\documentclass[11pt]{article}

\usepackage{listings}
\usepackage{epsfig}
\usepackage{lscape}
\usepackage{multirow}
\usepackage{longtable}
\usepackage{amsmath,amssymb,amsthm}
\usepackage{color}
\usepackage{placeins}
\usepackage{url}
\usepackage{cases}
\usepackage{hyperref}
\usepackage{setspace}
\usepackage{extramarks}
\usepackage{graphicx}
\usepackage{float}
\usepackage{subfig}
\usepackage{ctex}
\usepackage{algpseudocode}


\usepackage{booktabs}

\oddsidemargin 0pt
\evensidemargin 0pt
\marginparwidth 10pt
\marginparsep 10pt
\topmargin -20pt
\textwidth 6.5in
\textheight 8.5in
\parindent = 20pt

\DeclareMathOperator*{\argmin}{argmin}
\DeclareMathOperator*{\minimax}{minimax}

\renewcommand{\algorithmicrequire}{ \textbf{function:}}
\renewcommand{\algorithmicreturn}{ \textbf{end function}}
\newcommand{\blue}[1]{\begin{color}{blue}#1\end{color}}
\newcommand{\magenta}[1]{\begin{color}{magenta}#1\end{color}}
\newcommand{\red}[1]{\begin{color}{red}#1\end{color}}
\newcommand{\green}[1]{\begin{color}{green}#1\end{color}}


\newtheorem{theorem}{Theorem}
\newtheorem{proposition}{Proposition}
\newtheorem{lemma}{Lemma}
\newtheorem{corollary}{Corollary}
\newtheorem{remark}{Remark}
\newtheorem{assumption}{Assumption}
\newtheorem{definition}{Definition}
%\newenvironment{proof}{{\noindent\it Proof}\quad}{\hfill $\square$\par}

%\usepackage{sidecap}
\newcommand{\enterProblemHeader}[1]{
	\nobreak\extramarks{}{Problem \arabic{#1} continued on next page\ldots}\nobreak{}
	\nobreak\extramarks{Problem \arabic{#1} (continued)}{Problem \arabic{#1} continued on next page\ldots}\nobreak{}
}

\newcommand{\exitProblemHeader}[1]{
	\nobreak\extramarks{Problem \arabic{#1} (continued)}{Problem \arabic{#1} continued on next page\ldots}\nobreak{}
	\stepcounter{#1}
	\nobreak\extramarks{Problem \arabic{#1}}{}\nobreak{}
}

\setcounter{secnumdepth}{0}
\newcounter{partCounter}
\newcounter{homeworkProblemCounter}
\setcounter{homeworkProblemCounter}{1}
\nobreak\extramarks{Problem \arabic{homeworkProblemCounter}}{}\nobreak{}

%
% Homework Problem Environment
%
% This environment takes an optional argument. When given, it will adjust the
% problem counter. This is useful for when the problems given for your
% assignment aren't sequential. See the last 3 problems of this template for an
% example.
%
\newenvironment{homeworkProblem}[1][-1]{
	\ifnum#1>0
	\setcounter{homeworkProblemCounter}{#1}
	\fi
	\section{Problem \arabic{homeworkProblemCounter}}
	\setcounter{partCounter}{1}
	\enterProblemHeader{homeworkProblemCounter}
}{
	\exitProblemHeader{homeworkProblemCounter}
}

\begin{document}
	
	\title{\bf Homework 8}
	
	\author{陈远洋}
	
	
	\date{}
	\maketitle	
	
	\begin{homeworkProblem}
		Assume $C$ is a convex set, show that $x^*\in\argmin_{y\in C} \frac{1}{2}\|x-y\|_2^2$ if and only if for any $y\in C$, $\left\langle x-x^*, y-x^*\right\rangle\leq0$.
	\end{homeworkProblem}
    \begin{proof}
        先证明充分性:$x^*\in\argmin_{y\in C} \frac{1}{2}\|x-y\|_2^2\Rightarrow \forall y\in C\,, \left\langle x-x^*, y-x^*\right\rangle\leq0$

        若$x^*\in \argmin_{y\in C}\frac{1}{2}\|x-y\|_2^2$,则$\forall y \in C$,有:
        \begin{align*}
            \frac{1}{2}(\|x-y\|_2^2-\|x-x^*\|_2^2)&=\frac{1}{2}(\|x-x^*+x^*-y\|_2^2-\|x-x^*\|_2^2)\\
            &=\frac{1}{2}(\langle x^*-y,x^*-y\rangle-2\langle x-x^*,y-x^* \rangle)\\
            &\geq 0
        \end{align*}
        而若$\exists y_0\in C$s.t.$\langle x-x^*,y_0-x^*\rangle>0$,取$\alpha \in (0,1)$.由$C$是凸集且$x^*,y_0\in C$
        故$\forall \alpha\in (0,1):\, x^*+\alpha (y_0-x^*)\in C$
        令$g(\alpha)=\frac{1}{2}\|x-x^*+\alpha(y_0-x^*)\|_2^2-\frac{1}{2}\|x-x^*\|_2^2$,
        而进一步展开:
        \begin{align*}
            g(\alpha)&=\frac{\alpha^2}{2}\langle x^*-y_0,x^*-y_0\rangle-\alpha\langle x-x^*,y_0-x^* \rangle\\
            &=\frac{\alpha}{2}(\alpha \|x^*-y_0\|_2^2-2\langle x-x^*,y_0-x^*\rangle)
        \end{align*}
        则$\forall \alpha \in(0,1)\bigcap(0,\frac{\langle x-x^*,y_0-x^*\rangle}{\|x^*-y_0\|_2^2})\,:$
        $g(\alpha)<\frac{\alpha}{2}(\frac{\langle x-x^*,y_0-x^*\rangle}{\|x^*-y_0\|_2^2}\|x^*-y_0\|_2^2-2\langle x-x^*,y_0-x^*\rangle)=-\frac{\alpha}{2}\langle x-x^*,y_0-x^*\rangle<0$
        这与$x^*$是最优解的条件矛盾,故不存在这样的$y_0$。也即$\forall y\in C\,, \left\langle x-x^*, y-x^*\right\rangle\leq0$

        再来看必要性:$\forall y\in C\,, \left\langle x-x^*, y-x^*\right\rangle\leq0\Rightarrow x^*\in\argmin_{y\in C} \frac{1}{2}\|x-y\|_2^2$
        
        若$\forall y\in C\,:\langle x-x*,y-x^*\rangle\leq 0$且$\exists \bar{x}\neq x^*\in C\,: \|x-\bar{x}\|_2^2<\|x-x^*\|_2^2$(即$x^*$并非最优点)
        那么有:
        \begin{align*}
            \|x-\bar{x}\|_2^2-\|x-x^*\|_2^2&=\langle x-x^*+x^*-\bar{x}, x-x^*+x^*-\bar{x}\rangle-\langle x-x^*, x-x^*\rangle\\
            &=2\langle x-x^*, x^*-\bar{x}\rangle+\|\bar{x}-x^*\|_2^2\\
            &<0
        \end{align*}
        可得$\langle x-x^*, \bar{x}-x^*\rangle>\frac{1}{2}\|\bar{x}-x^*\|_2^2>0$矛盾!故不存在这样的$\bar{x}$也即
        $x^*\in \argmin_{y\in C}\frac{1}{2}\|x-y\|_2^2$.
    \end{proof}
	
	\begin{homeworkProblem}
		Use the optimality conditions to solve the following problem:
		\begin{equation*}
		\begin{aligned}
			& \min_{x\in\mathbb{R}^n} f(x) \\
			& s.t. \ h(x) = 0,
		\end{aligned}
		\end{equation*}
		where \\
		(a) $f(x)=\|x\|_2^2$, $h(x)=\sum_{i=1}^{n}x_i-1$.\\
		(b) $f(x)=\sum_{i=1}^{n}x_i$, $h(x)=\|x\|_2^2-1$.\\
		(c) $f(x)=\|x\|_2^2, h(x)=x^TQx-1$, where $Q$ is positive definite.
	\end{homeworkProblem}
    \begin{itemize}
        \item (a) 由一阶必要条件$\exists \lambda$,对$L(x;\lambda)=f(x)-\lambda h(x)$有:
        $\begin{cases}
            h(x)=\sum_{i=1}^{n}x_i-1=0&\\
            \nabla f(x)-\lambda \nabla h(x)=2x-\lambda \vec{1}=0&\\ 
        \end{cases}$
        联合两式解得$x_i=\frac{1}{n},i=1,2,\cdots n$.
        而$\nabla_{xx}^2L(x)=2I$,$\nabla h(x)=\vec{1}$,
        代入二阶充分条件$d^T\nabla_{xx}^2L(x)d>0,\forall d\neq0:\, d^T\nabla h(\vec{\frac{1}{n}})=0$
        成立,所以$\vec{\frac{1}{n}}$是最优解。对应函数值为$f(x)=\frac{1}{n}$

        \item (b) 由一阶必要条件$\exists \lambda$,对$L(x;\lambda)=f(x)-\lambda h(x)$有:
        $\begin{cases}
            h(x)=\langle x,x\rangle-1=0\\
            \nabla f(x)-\lambda \nabla h(x)=\vec{1}-2\lambda x=0
        \end{cases}$
        联合两式解得$x_i^2=\frac{1}{n},i=1,2,\cdots n$,且$x_1=\cdots=x_n$.
        而$\nabla_{xx}^2L(x)=-2\lambda I$,$\nabla h(x)=2x$,
        代入二阶充分条件$d^T\nabla_{xx}^2L(x)d>0,\forall d\neq0:\, d^T\nabla h(x)=0$.解得
        $\lambda<0$,则可得$x_i=\frac{-1}{\sqrt{n}}$为最优点。对应函数值为$f(\vec{\frac{-1}{\sqrt{n}}})=-\sqrt{n}$
        
        \item (c) 由一阶必要条件$\exists \lambda$,对$L(x;\lambda)=f(x)-\lambda h(x)$有:
        $\begin{cases}
            h(x)=xQ^Tx-1=0\\
            \nabla f(x)-\lambda \nabla h(x)=2x-2\lambda Qx=0
        \end{cases}$
        如果$\lambda=0$那么由第二个等式可推出$x=0$与第一式矛盾,所以$\lambda\neq0$
        所以原来的等式可以改写为:
         $\begin{cases}
            x^TQx=1\\
            (Q-\frac{1}{\lambda}I)x=0
        \end{cases}$
        可以看出,这是一个特征值问题。代入$f(x)$可得$f(x)=x^Tx=x^T\lambda Qx=\lambda$.
        而容易看出$\frac{1}{\lambda}$为$Q$的特征值。要想求出$f(x)$的最小值,我们需要$\frac{1}{\lambda}$尽可能大。
        于是取$\frac{1}{\lambda}=\|Q\|_2=\rho(Q)$,$x^*$为对应的模长为$\|Q\|_2^{-\frac{1}{2}}$特征向量。
        而$\nabla_{xx}^2L(x)=2(I-\lambda Q)$,$\nabla h(x)=2Qx$。
        而由于$d\bot span\{x^*\}$,$d^TQd<d^Td\|Q\|_2$,
        代入二阶充分条件$d^T\nabla_{xx}^2L(x^*)d>0,\forall d\neq0:\, d^T\nabla h(x^*)$
        成立.进一步推出该点为最优点。对应函数值为$f(x^*)=\frac{1}{\|Q\|_2}$
    \end{itemize}
	
	\begin{homeworkProblem}
		Consider a symmetric $n\times n$ matrix $Q$. Define
		\begin{equation*}
			\lambda_1 = \min_{\|x\|_2^2=1} x^TQx, \ e_1\in\argmin_{\|x\|_2^2=1} x^TQx,
		\end{equation*}
		and for $k=0,\cdots,n-1$,
		\begin{equation*}
			\lambda_{k+1}=\min _{\substack{\|x\|_2^2=1 \\ e_i^T x=0, i=1, \ldots, k}} x^T Q x, \quad e_{k+1} \in \arg \min _{\substack{\|x\|_2^2=1 \\ e_i^T x=0, i=1, \ldots, k}} x^T Q x
		\end{equation*}
		(a) Show that 
		$$
		\lambda_1\leq\lambda_2\leq\cdots\leq\lambda_n.
		$$
		(b) Show that the vectors $e_1,\cdots, e_n$ are linearly independent.\\
		(c) Interpret $\lambda_1,\lambda_2,\cdots,\lambda_n$ as Lagrange multipliers, and show that $\lambda_1,\lambda_2,\cdots,\lambda_n$ are the eigenvalues of $Q$, while $e_1,\cdots, e_n$ are corresponding eigenvectors.
	\end{homeworkProblem}
	\begin{proof}
        (a)
        从直观上来看,由于越往后约束更多。所以最优点函数值一定越来越大。
        作以下严格证明:如果$\exists k,\,$s.t. $\lambda_{k}>\lambda_{k+1}$
        那么$e_{k+1}^TQe_{k+1}=\lambda_{k+1}<\lambda_k$且:
        $\begin{cases}
            \|e_{k+1}\|_2=1\\
            e_j^Te_{k+1}=0&\forall j<k(k>1)            
        \end{cases}$
        即$e_{k+1}$是可行的。那么这与$\lambda_k$是子问题的最优解函数值矛盾!
        所以$\forall k,\lambda_{k}\leq\lambda_{k+1}$
        即$\lambda_1\leq\lambda_2\leq\cdots\leq\lambda_n.$

        (b)由第$k$步条件($k>1$):$e_i^Tx=0,\,\forall i<k$并且$e_{k}$是可行的
        所以$e_i^Te_k=0,\,\forall i<k$。如果$e_1,\cdots, e_n$不是线性无关。那么就$\exists k$,
        s.t.$e_k$可以由前$k-1$个向量线性表出(i.e.$k$可以取线性组合为0的最大不为0系数对应下标)。
        而这与正交性条件矛盾:$0<e_k^Te_k=e_k^T(\sum_{i=1}^{k-1}\alpha_i e_i)=0$.
        所以$e_1,\cdots, e_n$线性无关。

        (c)令$f(x)=x^TQx,\,h(x)=\|x\|_2^2-1=\langle x,x\rangle-1$
        定义问题1:
        \begin{align*}
            & \min_{x\in\mathbb{R}^n} f(x) \\
			& s.t. \ h(x) = 0    
        \end{align*}
        应用拉格朗日乘子$\lambda$,得到拉格朗日函数$L(x;\lambda)=f(x)-\lambda h(x)$.由一阶必要条件$\nabla f(x)-\lambda \nabla h(x)=2(Q-\lambda I)x=0$
        化简得到$(Q-\lambda I)x=0$左乘$x$得到$f(x)=\lambda$
        而这个等式是一个特征值问题,蕴含$\lambda,x$分别是Q的特征值与特征向量。
        此时$\nabla_{xx}^2 L=2(Q-\lambda I)$,
        代入二阶充分条件$d^T\nabla_{xx}^2Ld>0,\forall d\neq0: d^T\nabla h(x)=0$.
        显然此时$\lambda$取Q最小的特征值可以使得函数值最优。
        对应本问题我们得到$\lambda_1$与$e_1$是Q的特征值与特征向量。

        假设我们要证明的命题:$\lambda_i,e_i$是Q的第$i$个特征对在前$k$对成立,采用数学归纳法,下证$k+1$时的情况:
        在问题$k$的基础上定义问题$k+1$
        \begin{align*}
            & \min_{x\bot span\{e_1,\cdots,e_k\}} f(x) \\
			& s.t. \ h(x) = 0    
        \end{align*}
        引入乘子$\mu_1,\cdots,\mu_k,\lambda$,得到拉格朗日函数$L(x;\lambda;\mu)=f(x)-\lambda h(x)-\sum_{i=1}^{k}\mu_i x^Te_i$.
        由一阶必要条件$\nabla f(x)-\lambda \nabla h(x)-\sum_{i=1}^{k}\mu_i e_i=0$
        化简得到$(Q-\lambda I)x=\sum_{i=1}^{k}\frac{\mu_i}{2} e_i$
        分别左乘$e_i,i=1,\cdots,k$得到$e_i^TQx=\lambda_ie_i^Tx=0=\frac{\mu_i}{2}$.
        再左乘$x$得到$x^TQx=\lambda x^Tx=\lambda$
        由上面两个条件,必要条件进一步化简为:$(Q-\lambda I)x=0$而这同样是一个特征值问题,
        蕴含$\lambda,x$分别是Q的特征值与特征向量。
        由于已经求出$\mu_i=0$, 此时$\nabla_{xx}^2 L=2(Q-\lambda I)$,
        同样代入二阶充分条件$d^T\nabla_{xx}^2Ld>0,\forall d\neq0: d^T\nabla h(x)=0$,注意此时
        方向$d$的选取局限在前$k$个特征向量的补空间。
        显然此时$\lambda$取Q第$k+1$最小的特征值可以使得函数值最优。(由于前面已经取过前k小特征对了)
        故对应$\lambda_{k+1}$与$e_{k+1}$是Q的第$k+1$小的特征值与特征向量。

        综上所述,将$\lambda_i$作为Lagrange乘子的情况下,我们得到了$\lambda_1,\lambda_2,\cdots,\lambda_n$ 
        是 $Q$的特征值, 而 $e_1,\cdots, e_n$是对应特征向量。
    \end{proof}
	
\end{document}