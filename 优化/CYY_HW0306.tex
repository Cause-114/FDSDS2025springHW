\documentclass[11pt]{article}

\usepackage{listings}
\usepackage{epsfig}
\usepackage{lscape}
\usepackage{multirow}
\usepackage{longtable}
\usepackage{amsmath,amssymb,amsthm}
\usepackage{color}
\usepackage{placeins}
\usepackage{url}
\usepackage{cases}
\usepackage{hyperref}
\usepackage{setspace}
\usepackage{extramarks}
\usepackage{graphicx}
\usepackage{float}
\usepackage{subfig}
\usepackage{ctex}
\usepackage{algpseudocode}


\usepackage{booktabs}

\oddsidemargin 0pt
\evensidemargin 0pt
\marginparwidth 10pt
\marginparsep 10pt
\topmargin -20pt
\textwidth 6.5in
\textheight 8.5in
\parindent = 20pt

\DeclareMathOperator*{\argmin}{argmin}
\DeclareMathOperator*{\minimax}{minimax}

\renewcommand{\algorithmicrequire}{ \textbf{function:}}
\renewcommand{\algorithmicreturn}{ \textbf{end function}}
\newcommand{\blue}[1]{\begin{color}{blue}#1\end{color}}
\newcommand{\magenta}[1]{\begin{color}{magenta}#1\end{color}}
\newcommand{\red}[1]{\begin{color}{red}#1\end{color}}
\newcommand{\green}[1]{\begin{color}{green}#1\end{color}}


\newtheorem{theorem}{Theorem}
\newtheorem{proposition}{Proposition}
\newtheorem{lemma}{Lemma}
\newtheorem{corollary}{Corollary}
\newtheorem{remark}{Remark}
\newtheorem{assumption}{Assumption}
\newtheorem{definition}{Definition}
%\newenvironment{proof}{{\noindent\it Proof}\quad}{\hfill $\square$\par}

%\usepackage{sidecap}
\newcommand{\enterProblemHeader}[1]{
	\nobreak\extramarks{}{Problem \arabic{#1} continued on next page\ldots}\nobreak{}
	\nobreak\extramarks{Problem \arabic{#1} (continued)}{Problem \arabic{#1} continued on next page\ldots}\nobreak{}
}

\newcommand{\exitProblemHeader}[1]{
	\nobreak\extramarks{Problem \arabic{#1} (continued)}{Problem \arabic{#1} continued on next page\ldots}\nobreak{}
	\stepcounter{#1}
	\nobreak\extramarks{Problem \arabic{#1}}{}\nobreak{}
}

\setcounter{secnumdepth}{0}
\newcounter{partCounter}
\newcounter{homeworkProblemCounter}
\setcounter{homeworkProblemCounter}{1}
\nobreak\extramarks{Problem \arabic{homeworkProblemCounter}}{}\nobreak{}

%
% Homework Problem Environment
%
% This environment takes an optional argument. When given, it will adjust the
% problem counter. This is useful for when the problems given for your
% assignment aren't sequential. See the last 3 problems of this template for an
% example.
%
\newenvironment{homeworkProblem}[1][-1]{
	\ifnum#1>0
	\setcounter{homeworkProblemCounter}{#1}
	\fi
	\section{Problem \arabic{homeworkProblemCounter}}
	\setcounter{partCounter}{1}
	\enterProblemHeader{homeworkProblemCounter}
}{
	\exitProblemHeader{homeworkProblemCounter}
}

\begin{document}
	
	\title{\bf Homework 2}
	
	\author{ 陈远洋 
	}
	
	
	\date{}
	\maketitle
	
	\pagebreak
	
	 %问题一
	\begin{homeworkProblem}		
		Consider the least-squares problem 
		$$
		\min_x \frac{1}{2}\|Ax-b\|^2.
		$$
		Please proof that the problem always has at least one optimal solution. You may use the property of coercive functions.
	\end{homeworkProblem}
	
	
	%解答在这里写
	\begin{proof}
        \quad

        记$f(x)=\|Av\|^2,g(x)=\dfrac{1}{2}\|Ax-b\|^2$,并令
        $A=\begin{pmatrix}\vec{a_1}^T&\\\cdots&\\\vec{a_n}^T\end{pmatrix}
        \quad V=\text{span}(\vec{a_1},\cdots,\vec{a_n}) \quad
        B=V\bigcap \{\vec{v}\vert\|\vec{v}\|=1\}$
        \begin{enumerate}
        \item $V=\{\vec{0}\}$
                
        则显然$A=0_{n\times m},g(x)\equiv \dfrac{1}{2}\|b\|^2$,
        显然至少存在一个全局最小解。
        \item $V\neq \{\vec{0}\}$
        
        否则,B是一个有界闭集,故$f(v)=\|Av\|^2$在该取值集合上存在一个最小值,记为$t^2,t\geq0$.
        并且由于$v\in V$,故$t\neq 0$.
        而对于$\|Av-b\|\geq \|Av\|-\|b\|\geq \|v\|\|A\dfrac{v}{\|v\|}\|-\|b\|\geq t\|v\|-\|b\|$.
        
        $\forall v \in V\bigcap \{v\vert\|v\|> \dfrac{2\|b\|+1}{t}\}$.有
        $\|Av-b\|>\|b\|+1>0$.故对于$g(v)=\dfrac{1}{2}\|Av-b\|^2$,有
        $g(v)>\dfrac{1}{2}(\|b\|+1)^2> g(\vec{0})$.

        而对于$v\in V \bigcap\{v\vert\|v\|\leq\dfrac{2\|b\|+1}{t}\}$.可知自变量$v$取值为一个有界
        闭集,故一定存在一个最优解$v_0,s.t. $在该取值集合上$g(v_0)\leq g(v)$,固有$g(v_0)\leq g(\vec{0})$.

        而$\forall v \in V\bigcap \{v\vert\|v\|> \dfrac{2\|b\|+1}{t}\},g(v)>g(\vec{0})$.故$v_0$就是一个
        在$V$上的全局最优解。

        令$x=v_x+u_x$,其中$v_x\in V$,$u_x\in V^\bot$,则有:
        \[\|Ax-b\|^2=\|A(v_x+u_x)-b\|^2=\|Av_x-b\|^2\]
        故$g(x)=g(v_x+u_x)=g(v_x)$.由上面推导过程我们有:
        $\forall x \in \mathbb{R}^m,\forall u \in V^\bot, g(x)=g(v_x)\geq g(v_0)=g(v_0+u)$.
        也即$v_0+u$是一个全局最优解。
        \end{enumerate}
        综上所述,$g(x)$在$\mathbb{R}^m$上一定至少存在一个全局最优解。
    \end{proof}
\end{document}



% 若$V=\{\vec{0}\}$,显然$A=0_{m\times n}$,
% 那么$\forall x\in \mathbb{R}^n$,该式恒等于$\dfrac{1}{2}\|b\|^2$.
% 显然至少存在一个全局最小解(optimal solution).

% 否则根据$x \in V$以及$x \notin V$两种情况,我们分别作以下论述:
% \begin{enumerate}
%     \item $x \in V$:
%     令$t=\min{\|\vec{a_i}\|}, i=1,\cdots,n$,则有
%     \item $x \notin V$:
% \end{enumerate}