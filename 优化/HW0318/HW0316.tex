\documentclass[11pt]{article}

\usepackage{listings}
\usepackage{epsfig}
\usepackage{lscape}
\usepackage{multirow}
\usepackage{longtable}
\usepackage{amsmath,amssymb,amsthm}
\usepackage{color}
\usepackage{placeins}
\usepackage{url}
\usepackage{cases}
\usepackage{hyperref}
\usepackage{setspace}
\usepackage{extramarks}
\usepackage{graphicx}
\usepackage{float}
\usepackage{subfig}
\usepackage{ctex}
\usepackage{algpseudocode}
\usepackage{booktabs}
\usepackage{algorithm,algorithmicx,caption}

\oddsidemargin 0pt
\evensidemargin 0pt
\marginparwidth 10pt
\marginparsep 10pt
\topmargin -20pt
\textwidth 6.5in
\textheight 8.5in
\parindent = 20pt

\DeclareMathOperator*{\argmin}{argmin}
\DeclareMathOperator*{\minimax}{minimax}

\renewcommand{\algorithmicrequire}{ \textbf{function:}}
\renewcommand{\algorithmicreturn}{ \textbf{end function}}
\newcommand{\blue}[1]{\begin{color}{blue}#1\end{color}}
\newcommand{\magenta}[1]{\begin{color}{magenta}#1\end{color}}
\newcommand{\red}[1]{\begin{color}{red}#1\end{color}}
\newcommand{\green}[1]{\begin{color}{green}#1\end{color}}


\newtheorem{theorem}{Theorem}
\newtheorem{proposition}{Proposition}
\newtheorem{lemma}{Lemma}
\newtheorem{corollary}{Corollary}
\newtheorem{remark}{Remark}
\newtheorem{assumption}{Assumption}
\newtheorem{definition}{Definition}
%\newenvironment{proof}{{\noindent\it Proof}\quad}{\hfill $\square$\par}

%\usepackage{sidecap}
\newcommand{\enterProblemHeader}[1]{
	\nobreak\extramarks{}{Problem \arabic{#1} continued on next page\ldots}\nobreak{}
	\nobreak\extramarks{Problem \arabic{#1} (continued)}{Problem \arabic{#1} continued on next page\ldots}\nobreak{}
}

\newcommand{\exitProblemHeader}[1]{
	\nobreak\extramarks{Problem \arabic{#1} (continued)}{Problem \arabic{#1} continued on next page\ldots}\nobreak{}
	\stepcounter{#1}
	\nobreak\extramarks{Problem \arabic{#1}}{}\nobreak{}
}

\setcounter{secnumdepth}{0}
\newcounter{partCounter}
\newcounter{homeworkProblemCounter}
\setcounter{homeworkProblemCounter}{1}
\nobreak\extramarks{Problem \arabic{homeworkProblemCounter}}{}\nobreak{}

%
% Homework Problem Environment
%
% This environment takes an optional argument. When given, it will adjust the
% problem counter. This is useful for when the problems given for your
% assignment aren't sequential. See the last 3 problems of this template for an
% example.
%
\newenvironment{homeworkProblem}[1][-1]{
	\ifnum#1>0
	\setcounter{homeworkProblemCounter}{#1}
	\fi
	\section{Problem \arabic{homeworkProblemCounter}}
	\setcounter{partCounter}{1}
	\enterProblemHeader{homeworkProblemCounter}
}{
	\exitProblemHeader{homeworkProblemCounter}
}

\begin{document}
	
	\title{\bf Homework 3}
	
	\author{ 陈远洋 \quad 23307130322 
	}
	
	
	\date{\today}
	\maketitle
	
	\pagebreak
	
	 %问题一
	\begin{homeworkProblem}		
		Consider the smoothed LASSO (Least Absolute Shrinkage and Selection Operator) problem: 
		$$
		\min_x \frac{1}{2}\|Ax-b\|_2^2 + \mu L_{\sigma}(x),
		$$
		where $x\in\mathbb{R}^n, A\in\mathbb{R}^{m\times n}$, $b\in\mathbb{R}^m$, $\mu$ is the regularization parameter. $L_{\sigma}(x)$ is defined as
		$$
		L_{\sigma}(x) = \sum_{i=1}^{n}l_{\sigma}(x_i),
		$$
		$$
		l_{\sigma}(x_i) = \left\{
		\begin{aligned}
		& \frac{1}{2\sigma}x_i^2, \ |x_i|<\sigma, \\
		& |x_i|-\frac{\sigma}{2}, \ \text{otherwise}.
		\end{aligned}
		\right.
		$$
		
		Let $\sigma=0.1$. Please use a gradient descent method to numerically solve above problem with $A$ and $b$ provided in the zip file, where $m=512, n=1024$. Then write a report to illustrate the method you used and present your experiment results. \\
		
		Note: 
		\begin{itemize}
			\item The regularization parameter $\mu$ can be small, e.g. $\mu=1\times10^{-2}$. It can be adjusted by yourself.
		\end{itemize}
		
	\end{homeworkProblem}
	
	\section{Solution}
    \subsection{解存在性分析}
    首先,该问题一定是有解的。类似于上一次的作业,我们只讨论$x\in A$的零空间的正交补情况(其他情况可以转化)。
    记$t^2=min_{\|x\|_2=1}\|Ax-b\|_2^2$,则有当$\|x\|_2\geq \dfrac{2\|b\|_2+1}{t}$时。
    由于$\dfrac{\|x_i\|_2^2}{2\sigma}-\|x_i\|_2+\dfrac{\sigma}{2}=\dfrac{1}{2\sigma}(\|x_i\|_2-\sigma)^2\geq0$。
    且$\|Ax-b\|_2\geq\|Ax\|_2-\|b\|_2\geq t\|x\|_2-\|b\|_2\geq0$,此时有:原式$\geq\dfrac{1}{2}(\|x\|_2t-\|b\|_2)^2+\mu(\sum_{i=1}^{n}|x_i|-\dfrac{n\sigma}{2})$
    $\geq\dfrac{1}{2}t^2\|x\|_2^2+(-2tb+\mu)\|x\|_2+O(1)$.这是一个关于$\|x\|_2$的二次函数,可知在$\|x\|_2\rightarrow \infty$时,原式一定也区域无穷。
    又由于在$\|x_i\|_2=\sigma$处,两个分段函数均趋近于$\dfrac{\sigma}{2}$,故原式在小范围内连续。
    所以原式一定有全局最优解。
    \subsection{代码构成分析}
    梯度下降法的几个步骤:
    \begin{algorithm}[!htb]\label{alg1}
    \caption{Gradient Descent}
    \begin{algorithmic}[1]
    \State Input $f$,$x_0$,$tolerance$,
    \State $k\leftarrow 0$
    \While {$\nabla f(x_k)\geq tolerance$}
    \State pick a descent direction $p_k$ such that $\nabla f(x_k)^Tp_k<0$.
    \State pick a step size $\alpha_k$.
    \State update $x_k$ : $x_{k+1}\leftarrow x_k+\alpha_kp_k$.
    \State $k\leftarrow k+1$;
    \EndWhile
    \end{algorithmic}
    \end{algorithm}
    其中目标函数为LASSO问题中定义的函数,在我的代码中实现在最后,名字为:$res=lasso(A,b,\sigma,\mu,x)$
    对应的梯度函数为$dfres=dflasso(A,b,\sigma,\mu,x)$。由于问题关心的最优解是$x$,所以我在代码文件的最
    前面将目标函数与梯度函数定义为$f$,$df$两个关于$x$单一变量的函数句柄(带入了预设值得参数值,可以改变)。

    而下降方向的选择上,由于缺乏进一步的学习,我直接采用了负梯度方向。由于寻找下降方向采用的函数封装连起来,
    后续期望可以用其他下降方向更改。

    对于线搜索产生的步长上,我采用了两种方法,一种是armijo,另一种是wolfe。
    原始代码如下:
    \begin{lstlisting}[language=Matlab,frame=single,numbers=left]
    function a = armijo(f, df, d, x, a0)
        a = a0; f0 = f(x); df0 = dot(df(x), d);
        c1 = 0.085; beta = 0.35; fx = f(x + a * d);
        while (fx > f0 + c1 * a * df0)
            a = a * beta; fx = f(x + a * d);
            if (a < 1e-10 && fx <= f0)
                break;
            end
        end
    end

    function a = wolfe(f, df, d, x, a0)
        a = a0; c1 = 0.09; c2 = 0.3;
        f0 = f(x); df0 = dot(df(x), d);
        secant = (f(x + a * d) - f0) / a; dfx = dot(df(x + a * d), d);
        while (secant > df0 * c1 || dfx < df0 * c2)
            % use the quadratic interpolation (f(0),f'(0),f(a)).
            % update a as the minimum point of quadratic function.
            a = df0 * a / (2 * (df0 - secant));
            secant = (f(x + a * d) - f0) / a; 
            dfx = dot(df(x + a * d), d);
            if (a < 1e-6 && secant <= 0)
                break;
            end
        end
    end
    \end{lstlisting}
    
    其中armijo函数采用armijo准则判断步长是否合适,即:$f(x+\alpha d)\leq f(x)+c_1\alpha\nabla f(x)^Td$。
    当步长不满足条件时,指数衰减步长,每次步长乘以$\beta$。直到满足条件,同时为了保证在步长较小时
    舍入误差导致$f(x+\alpha d)$与$f(x)$相等,增加了判断条件$\alpha<1e-10\bigwedge f(x+\alpha d)\leq f(x)$
    满足时直接跳出。
    
    wolfe函数采用wolfe准则判断步长是否合适,即:$f(x+\alpha d)\leq f(x)+c_1\alpha\nabla f(x)^Td$(1)
    并且有$\nabla f(x+\alpha d)\geq c_2\nabla f(x)$。当步长不满足条件时,采用二次插值法更新步长。
    记$f(x+\alpha d)=\phi(\alpha)$即利用$\phi(0),\phi'(0),\phi(a)$三点确定二次函数的最小值,更新步长为插值函数的最小值点。
    让我们来进行计算差分:\[\phi[0]=\phi(0),\phi[\alpha]=\phi(\alpha)\]
    \[\phi[0,0]=\phi'(0),\, \phi[0,\alpha]=\dfrac{\phi(\alpha)-\phi(0)}{\alpha}\]
    \[\phi[0,0,\alpha]=\dfrac{\phi[0,\alpha]-\phi[0,0]}{\alpha}\]
    令$secant=\dfrac{f(x+\alpha d)-f(x)}{\alpha}$,则有:(1)$\Leftrightarrow secant\leq c_1\nabla f(x)^Td$ 
    且二次函数最低点$\alpha_{min}$满足$\alpha_{min}=\dfrac{-\phi[0,0]}{2\phi[0,0,\alpha]}=\dfrac{-\nabla f(x)^Td \alpha}{2(secant - \nabla f(x)^Td)}$。
    进一步地,为了防止舍入误差导致更新步长与函数值相差过小,增加了判断条件$\alpha<1e-6\bigwedge secant\leq 0$,满足时直接跳出。

    整体上,将线搜索、方向选择函数实现封装为$ls$与$dir$句柄,供主要的梯度下降函数$gradient_descent$调用。
    通过改变句柄对应的函数可以更换线搜索以及方向选择方法。设置梯度精度要求以及最大迭代次数用作停机条件。整个$gradient_descent$函数的实现
    逻辑与\hyperref[alg1]{算法一}一致,这里不再赘述。

    \subsection{求解过程}
    首先,我们导入数据$A,b$并设置参数$\sigma=0.1,\mu=1\times10^{-2}$。
    并设置初始值步长$\alpha_0$,解$x_0=\vec{0}$。
    根据需求,设置线搜索方法以及方向选择方法(更改$ls,dir$句柄即可)。
    然后设置最大迭代次数与梯度精度要求,静待结果。

    实际调整参数的过程是非常漫长的,但需要调整的主要是线搜索方法内部的参数$c_1,c_2,\beta$以及
    步长初始值$\alpha_0$。经过多次尝试,我最终选择了如下参数:
    \begin{itemize}
        \item armijo, $c_1=0.085,beta=0.35$
        \item wolfe, $c_1=0.09,c_2=0.3$
    \end{itemize}
    步长初始值$\alpha_0=1\times10^{-3}$。

    虽然二次插值的方法任然不能保证一定能找到满足$wolfe$准则的步长,但在我的实验中,
    这种情况没有发生。而对于那些找到后步长已经非常小甚至使得函数值
    几乎不变的情况,我则直接跳出循环,认为已经找到了最优解(这种情况在设置梯度模长
    要求为1e-6时发生过,如果不直接跳出,则会陷入无限循环)。
    \subsection{实验结果}
    以题设$\mu=1\times10^{-2}$,为标准情况。最优解$x$存放在solution.mat文件中.
    $f(x)=0.818310$,$\|\nabla f(x)\|=1.5805\times 10^{-6}$。
    
    而不同的方法选择,精度对实验结果有着显著影响。结果如下:
    \begin{table}[!htb]
    \centering
    \caption{不同方法选择,精度对结果的影响}
    \begin{tabular}{c|c|c|c|c|c}
    \hline
    方法 & 精度 & 耗时 & 迭代次数 & 函数值&梯度模长 \\ \hline
    wolfe & 1e-4 & 70.97s & 251571 & 0.818310 & 9.9998e-5 \\ \hline
    wolfe & 1e-5 & 371.49s & 333266 & 0.818310 & 9.9971e-6 \\ \hline
    wolfe & 1e-6 & 429.27s & 409115 & 0.818310 & 1.6687e-6\\ \hline
    armijo & 1e-3 & 49.21s & 255715 & 0.818329 & 9.9889e-4 \\ \hline
    armijo & 1e-4 & 248.27s & 361530 & 0.818310 & 9.9930e-5 \\ \hline
    armijo & 1e-5 & 94.55s & 476582 & 0.818310 & 9.9984e-6 \\ \hline
    \end{tabular}
    \end{table}

    与此同时,当我们改变$\mu$时,结果、收敛速度会显著变化。
    具体来说,当我们统一要求梯度模长在1e-5以下时,我们有以下结果:
    \begin{table}[!htb]
    \centering
    \caption{不同$\mu$选择对结果的影响}
    \begin{tabular}{c|c|c|c|c}
        \hline
        $\mu$ & 耗时 & 迭代次数 & 函数值 & 梯度模长\\ \hline
        1e0 &1.44s& 4733&81.643264&4.6859e-6\\ \hline
        1e-1 & 11.09s & 41476 & 8.181363 & 9.9220e-6 \\ \hline
        1e-2 & 98.46s & 333266 & 0.818310& 9.9971e-6 \\ \hline
        1e-3& 327.30s & 1000000(超界未停)& 0.082180&0.003342 \\ \hline
        0 & 0.11s & 141 & 3.1684e-13 &8.0594e-6\\ \hline
    \end{tabular}
    \end{table}

    可以看到,一定范围内随着$\mu$的减小,算法收敛速度变慢,但是当$\mu\rightarrow 0$时,
    算法将飞速收敛。

    值得注意的是,耗时受到硬件以及设备运行状态等多方面影响,比如在上面两个表格中,wolfe准则在$\mu=1e-5$
    的运行时间相差较大。故其仅具有相对的参考价值。
    
    以迭代次数为横轴,函数值与最优值的差的以10为底的对数为纵轴,绘制出二者关系曲线。如下图所示:
    \begin{minipage}[!htb]{0.45\textwidth}
        \includegraphics[width=\textwidth]{wolfe_mu1e0.png}
        \captionof{figure}{$\mu=1$时的收敛历史}
    \end{minipage}\hfill
    \begin{minipage}[!htb]{0.45\textwidth}
        \includegraphics[width=\textwidth]{wolfe_mu1e-1.png}
        \captionof{figure}{$\mu=1e-1$时的收敛历史}
    \end{minipage}
    \begin{minipage}[!htb]{0.45\textwidth}
        \includegraphics[width=\textwidth]{wolfe_mu1e-2.png}
        \captionof{figure}{$\mu=1e-2$时的收敛历史}
    \end{minipage}\hfill
    \begin{minipage}[!htb]{0.45\textwidth}
        \includegraphics[width=\textwidth]{wolfe_mu1e-3.png}
        \captionof{figure}{$\mu=1e-3$时的收敛历史}
    \end{minipage}
    \begin{minipage}[!htb]{0.45\textwidth}
        \includegraphics[width=\textwidth]{wolfe_mu0.png}
        \captionof{figure}{$\mu=0$时的收敛历史}
    \end{minipage}
    \subsection{更新}
    在上过第二周的课后,我对此次作业有如下更新:
    \begin{enumerate}
        \item 增加了对收敛性的分析。
        \item 对armijo准则的参数选择进行了优化,使得迭代次数进一步减小。
    \end{enumerate}
    
    首先对本问题。对梯度$\nabla f(x)=A^T(Ax-b)+\mu h(x)$,其中$h(x)_i=\begin{cases}
        & \frac{x_i}{\sigma}, \ |x_i|<\sigma, \\
		& sign(x_i), \ \text{otherwise}.
    \end{cases}$
    在迭代格式$x_{k+1}=x_k-\alpha_k\nabla f(x_k)$下。有:\begin{align*}
        x_{k+1}-x_*&=x_k-x_*-\alpha_k(A^T(Ax_k-b)-\mu h(x_k))\\
            &= x_k-x_*-\alpha_k(\nabla f(x_*)+A^TA(x_k-x_*)+\mu(h(x_k)-h(x_*)))\\
            &=(I-\alpha_k A^TA)(x_k-x_*)-\alpha\mu(h(x_k)-h(x_*))
    \end{align*}
    并且在$\|x_k-x_*\|$很小时,我们可以近似的认为$h(x_k)\approx h(x_*)$,其中$x_*$指代最优解。
    因此,$\|x_{k+1}-x_*\|\leq max\{|1-\alpha\lambda_n|,|1-\alpha\lambda_1|\}\|x_k-x_*\|$, 其中$\lambda_n, \lambda_1$是
    矩阵$A^TA$的两个最大、最小特征值。在本问题中由于$A$是一个行数为列数一半的矩阵,所以$A^TA$至少有一半的为0的特征值,
    而经过一番计算可知$A^TA$的特征值中恰有512个零,且$\lambda_{513}=89.8163$, $\lambda_{1024}=2937.7$.
    故问题本身较为病态。
    
    在步长的选择上,对我上面的$wolfe$准则下搜索的步长取对数下来看,可以发现较多情况下步长
    均在$10^{-3}$左右。带入上式,由于有一半的0特征值,故很大概率下前面的二范数项几乎不起作用,迭代前进主要由后面的
    “一范数”惩罚项决定。由于在$|x_i|\geq\sigma$时,$h(x_i)$对相同符号下的值恒定。故我们考虑$|x_i|<\sigma$
    的情况能让$\Delta h(x)$更大,下降更多,而此时有$x_{k+1}-x_*=(1-\dfrac{\alpha\mu}{\sigma})(x_k-x_*)$
    可以看到此时下降的速度$1-\dfrac{\alpha\mu}{\sigma}$量级为$1-10^{-4}$。所以下降速度十分糟糕。
    估计下$(1-10^{-4})^{10^4}\approx \dfrac{1}{e}$,而$log(10^5)\approx 11.5$。故需要$11.5\times 10^4$次迭代
    才能达到$10^{-5}$精度。
	
    那么$10^{-3}$的步长有多糟糕呢?我们来计算一下全局的Lipschitz常数$L$,
    容易得到$L=\lambda_{1024}+\dfrac{\mu}{\sigma}$。对应的全局收敛区间为$[0,\dfrac{2}{L}]$约为$[0,6.8\times 10^{-4}]$
    可以看到,这个步长已经十分糟糕了。

    所以我们希望尽量选取较大的步长,那么经过我又几个晚上的调参,最终
    我能够使得在$\mu=1\times10^{-2}$时,平均步长上升至$10^{-2.5}$左右,
    这个时候的收敛次数就较少了。当然,我觉得这个问题差不多就是这样了,因为它实在是一个较为
    病态的问题。

    最终代码放在“solution1.m”中,求解结果如下图所示:
    
    \begin{minipage}[!htb]{0.45\textwidth}
        \includegraphics[width=\textwidth]{armijomu1e-2.png}
        \captionof{figure}{$\mu=1\times10^{-2}$,函数值减去最小值取对数的收敛历史}
    \end{minipage}\hfill
    \begin{minipage}[!htb]{0.45\textwidth}
        \includegraphics[width=\textwidth]{step_size.png}
        \captionof{figure}{$\mu=1\times10^{-3}$,步长取对数的收敛历史}
    \end{minipage}
    \begin{figure}
        \centering
        \includegraphics[width=0.8\textwidth]{res.png}
        \caption{$\mu=1\times10^{-2}$,梯度模长达到不同阶段时的效果}
    \end{figure}
\end{document}