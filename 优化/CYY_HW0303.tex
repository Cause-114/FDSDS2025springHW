\documentclass[11pt]{article}

\usepackage{listings}
\usepackage{epsfig}
\usepackage{lscape}
\usepackage{multirow}
\usepackage{longtable}
\usepackage{amsmath,amssymb,amsthm}
\usepackage{color}
\usepackage{placeins}
\usepackage{url}
\usepackage{cases}
\usepackage{hyperref}
\usepackage{setspace}
\usepackage{extramarks}
\usepackage{graphicx}
\usepackage{float}
\usepackage{subfig}
\usepackage{ctex}
\usepackage{algpseudocode}


\usepackage{booktabs}

\oddsidemargin 0pt
\evensidemargin 0pt
\marginparwidth 10pt
\marginparsep 10pt
\topmargin -20pt
\textwidth 6.5in
\textheight 8.5in
\parindent = 20pt

\DeclareMathOperator*{\argmin}{argmin}
\DeclareMathOperator*{\minimax}{minimax}

\renewcommand{\algorithmicrequire}{ \textbf{function:}}
\renewcommand{\algorithmicreturn}{ \textbf{end function}}
\newcommand{\blue}[1]{\begin{color}{blue}#1\end{color}}
\newcommand{\magenta}[1]{\begin{color}{magenta}#1\end{color}}
\newcommand{\red}[1]{\begin{color}{red}#1\end{color}}
\newcommand{\green}[1]{\begin{color}{green}#1\end{color}}


\newtheorem{theorem}{Theorem}
\newtheorem{proposition}{Proposition}
\newtheorem{lemma}{Lemma}
\newtheorem{corollary}{Corollary}
\newtheorem{remark}{Remark}
\newtheorem{assumption}{Assumption}
\newtheorem{definition}{Definition}
%\newenvironment{proof}{{\noindent\it Proof}\quad}{\hfill $\square$\par}

%\usepackage{sidecap}
\newcommand{\enterProblemHeader}[1]{
	\nobreak\extramarks{}{Problem \arabic{#1} continued on next page\ldots}\nobreak{}
	\nobreak\extramarks{Problem \arabic{#1} (continued)}{Problem \arabic{#1} continued on next page\ldots}\nobreak{}
}

\newcommand{\exitProblemHeader}[1]{
	\nobreak\extramarks{Problem \arabic{#1} (continued)}{Problem \arabic{#1} continued on next page\ldots}\nobreak{}
	\stepcounter{#1}
	\nobreak\extramarks{Problem \arabic{#1}}{}\nobreak{}
}

\setcounter{secnumdepth}{0}
\newcounter{partCounter}
\newcounter{homeworkProblemCounter}
\setcounter{homeworkProblemCounter}{1}
\nobreak\extramarks{Problem \arabic{homeworkProblemCounter}}{}\nobreak{}

%
% Homework Problem Environment
%
% This environment takes an optional argument. When given, it will adjust the
% problem counter. This is useful for when the problems given for your
% assignment aren't sequential. See the last 3 problems of this template for an
% example.
%
\newenvironment{homeworkProblem}[1][-1]{
	\ifnum#1>0
	\setcounter{homeworkProblemCounter}{#1}
	\fi
	\section{Problem \arabic{homeworkProblemCounter}}
	\setcounter{partCounter}{1}
	\enterProblemHeader{homeworkProblemCounter}
}{
	\exitProblemHeader{homeworkProblemCounter}
}

\begin{document}

	\date{}
	\title{\bf Homework 1}
	\author{ 陈远洋}
	\maketitle
	
	\pagebreak
	
	 %问题一
	\begin{homeworkProblem}		
		Let $F:\mathbb{R}^{n\times n}\to\mathbb{R}^{n\times n}$ and $F(X) = X^{-1}$ for any invertible matrix $X\in\mathbb{R}^{n\times n}$. $F$ is differentiable at $X$ indicates that given any perturbation matrix $\Delta X\in\mathbb{R}^{n\times n}$ such that $X+\Delta X$ is also invertible, it holds that $F(X+\Delta X) = F(X) + A(\Delta X) + o\left(\|\Delta X\|\right)$. Please show the form of $A(\Delta X)$. 
	\end{homeworkProblem}
	
    \textbf{Solution:}

    Firstly, let us to get a formula for $(I+\Delta X)^{-1}$, in which $\|\Delta X\|_F <1$. We have:
    \begin{equation}
        f(X,n)=I-\Delta X+\Delta X^2+\cdots+(-\Delta X)^{n}
    \end{equation}
    And we can get: 
    \begin{equation}
        f(X,n)(I+\Delta X)=I-\Delta X + \Delta X + \cdots - (-\Delta X)^{n+1}=I-(-\Delta X)^{n+1}       
    \end{equation}
    when $\lim_{n}=+\infty$, the addional term $\Delta X^{n+1}$ will be negligible. 
    Because that $\|(-\Delta X)^{n+1}\|_F \leq \|\Delta X\|_F^{n+1}$, when $\lim_{n}=\infty$, 
    obviously, it is zero. Therefore, we can get(known as the neumann series):
    \begin{equation}
        (I+\Delta X)^{-1}=\lim_{n\to\infty}\sum_{i=0}^{n}(-\Delta X)^{i}
    \end{equation}
    So we can get: 
    \begin{equation}
        F(X+\Delta X) = (X(I+X^{-1}\Delta X))^{-1}=(I+X^{-1}\Delta X)^{-1}X^{-1}=(\lim_{n\to\infty}\sum_{i=0}^{n}(-X^{-1}\Delta X)^{i})X^{-1}
    \end{equation}
    For that the first n term's degree is lower than n, as for the problem request, we get the first two terms,that are:
    \begin{equation}
        F(X+\Delta X)= X^{-1} - X^{-1}\Delta X X^{-1}+o(\|\Delta X\|^{-1})=F(X)-X^{-1}\Delta X X^{-1}+o(\|\Delta X\|)
    \end{equation}
    That is the answer form for the problem. Which means that the $A(\Delta X) = -X^{-1}\Delta X X^{-1}$ .
    %解答在这里写
	%\begin{}
		
	%\end{}
	
\end{document}