\documentclass[11pt]{article}
\usepackage{amsmath,amssymb,amsthm}
\usepackage{algorithm,algorithmicx}
\usepackage{algpseudocode}
\usepackage{ctex}
\usepackage{extramarks}
\usepackage{hyperref}
\usepackage{listings}


\oddsidemargin 0pt
\evensidemargin 0pt
\marginparwidth 10pt
\marginparsep 10pt
\topmargin -20pt
\textwidth 6.5in
\textheight 8.5in
\parindent = 20pt

\DeclareMathOperator*{\argmin}{argmin}
\DeclareMathOperator*{\minimax}{minimax}

\newtheorem{theorem}{Theorem}
\newtheorem{lemma}{Lemma}


\newcommand{\enterProblemHeader}[1]{
	\nobreak\extramarks{}{Problem \arabic{#1} continued on next page\ldots}\nobreak{}
	\nobreak\extramarks{Problem \arabic{#1} (continued)}{Problem \arabic{#1} continued on next page\ldots}\nobreak{}
}

\newcommand{\exitProblemHeader}[1]{
	\nobreak\extramarks{Problem \arabic{#1} (continued)}{Problem \arabic{#1} continued on next page\ldots}\nobreak{}
	\stepcounter{#1}
	\nobreak\extramarks{Problem \arabic{#1}}{}\nobreak{}
}

\setcounter{secnumdepth}{0}
\newcounter{partCounter}
\newcounter{homeworkProblemCounter}
\setcounter{homeworkProblemCounter}{1}
\nobreak\extramarks{Problem \arabic{homeworkProblemCounter}}{}\nobreak{}

\newenvironment{homeworkProblem}[1][-1]{
	\ifnum#1>0
	\setcounter{homeworkProblemCounter}{#1}
	\fi
	\section{Problem \arabic{homeworkProblemCounter}}
	\setcounter{partCounter}{1}
	\enterProblemHeader{homeworkProblemCounter}
}{
	\exitProblemHeader{homeworkProblemCounter}
}

\begin{document}
	
	\title{\bf Homework 7}
	\author{ 陈远洋 }
	\date{}
	\maketitle	
	
	\begin{homeworkProblem}
		Consider the problem
		\begin{equation*}
		\begin{aligned}
			& \max_{x_1,\cdots,x_n} x_1^{a_1}x_2^{a_2}\cdots x_n^{a_n} \\
			& \text{s.t.} \sum_{i=1}^{n} x_i = 1, \ x_i\geq 0, i=1,\cdots,n,
		\end{aligned}
		\end{equation*}
		where $a_i$ are given positive scalars. Find a global maximum and show that it is unique.
	\end{homeworkProblem}
	 \subsection{Solution}
    令$\beta_i=\frac{a_i}{\sum_{i=1}^{n} a_i},\,y_i=\frac{x_i}{\beta_i},\,Q=x_1^{a_1}x_2^{a_2}\cdots x_n^{a_n}$。\\则有
    $\ln Q=\sum_{i=1}^{n}a_i\ln x_i=(\sum_{i=1}^{n}a_i)(\sum_{i=1}^{n} \beta_i\ln y_i+\sum_{i=1}^{n}\beta_i\ln\beta_i)$。
    所以要使得Q最大,只需使得$\sum_{i=1}^{n}\beta_i\ln y_i$最大。而由Jensen不等式可知:
    \begin{align*}
        \sum_{i=1}^{n} \beta_i\ln y_i &\leq \ln(\sum_{i=1}^{n} \beta_iy_i) \\
        &=\ln (\sum_{i=1}^{n} x_i)\\
        &=0\\
    \end{align*}
    当且仅当$y_1=\cdots=y_n=1$, 即$x_i=\frac{a_i}{\sum_{i=1}^{n} a_i}$时等号成立。此时$Q$最大可取得$\prod_{i=1}^{n}(\frac{a_i}{\sum_{i=1}^{n} a_i})^{a_i}$。

    其实由Jensen不等式已经可知上面取值的唯一性。但是下面我们对此严加证明。

    首先,我们考虑$g(x)=\ln x$的凹凸性:我们给出以下引理:
    \begin{theorem}
        $-g(x)$在$\mathbb{R^+}$上是强凸的,即$\forall x\neq y \in \mathbb{R^+},\forall \lambda\in(0,1),\ g(\lambda x+(1-\lambda)y)>\lambda g(x)+(1-\lambda)g(y)$。
    \end{theorem}

    \begin{proof}
        固定$y\in\mathbb{R^+},\, \lambda\in(0,1)$。令$f(x)=\lambda g(x)+(1-\lambda)g(y)-g(\lambda x+(1-\lambda)y)$。
        容易看出$f(y)=0$。对$f(x)$求导,有$f'(x)=\lambda g'(x)+\lambda g'(\lambda x+(1-\lambda)y)=\lambda(\frac{1}{x}-\frac{1}{\lambda x+(1-\lambda)y})$。
        可知$\forall 0<x<y,\,f'(x)>0;\,\forall x>y,\,f'(x)<0$。所以f(x)在$(0,y)$严格单调递增,在$(y,+\infty)$严格单调递减。
        由此可知$f(x)<0,\,\forall x\neq y\in\mathbb{R^+}$。
    \end{proof}

    进一步由于$h(X)=\ln Q=\sum_{i=1}^{n}a_i\ln x_i$是若干个关于$x_i$的强凹函数凸组合,所以$h(X)$也是强凹函数。
    所以如果$\exists X_1\neq X_2 \in \mathbb{R^{\text{n+}}},\,s.t.\, h(X_1)=h(X_2)$并且二者都是最优解。
    那么$\forall \lambda\in(0,1),\, h(\lambda X_1+(1-\lambda)X_2)>\lambda h(X_1)+(1-\lambda)h(X_2)=h(X_1)$。
    此时与最优解的条件矛盾,所以最优解一定是唯一的。

    tips: 这题如果不用琴声不等式的话也可以用拉格朗日乘子法,对原函数取对数后加上$\lambda \sum x_i$。再对$x_i$求导
    得到$\frac{a_i}{x_i}-\lambda=0$,而$\sum x_i=1$,所以$x_i=\frac{a_i}{\sum_{i=1}^{n} a_i}$。


	\begin{homeworkProblem}
		Show that if $x^*$ is a local minimum of the twice continuously differentiable function $f:\mathbb{R}^n \mapsto \mathbb{R}$ over the convex set $X$, then
		\begin{equation*}
			(x-x^*)^T\nabla^2f(x^*)(x-x^*) \geq 0, 
		\end{equation*}
		for all $x\in X$, such that $\nabla f(x^*)^T(x-x^*) = 0$.
	\end{homeworkProblem}
    \subsection{Solution}
    采用反证法,假设这不是必要的。即有$x^*$是凸集$X$二阶连续可微函数$f:\mathbb{R}^n \mapsto \mathbb{R}$的局部最小值,
    并且$\exists x\in X, \nabla f(x^*)^T(x-x^*)=0$,且有$t=(x-x^*)^T\nabla^2f(x^*)(x-x^*)<0$。\\
    令$g(h)=f(x^*+h(x-x^*)),\,h\in [0,1]$。由X是凸集可知$\forall h\in [0,1],\, x^*+h(x-x^*)\in X$。
    又因为$f$二阶连续可微,所以$g$也是二阶连续可微。将$g(h)$对$h$在0点二阶泰勒展开,有
    \begin{align*}
        g(h) &= g(0)+g'(0)h+g''(0)\frac{h^2}{2}+o(h^2)\\
        &=f(x^*)+\nabla f(x^*)^T(x-x^*)h+(x-x^*)^T\nabla^2f(x^*)(x-x^*)\frac{h^2}{2}+o(h^2)\\
        &=f(x^*)+t\frac{h^2}{2}+o(h^2)\\
        &=f(x^*)+\frac{h^2}{2}(t+o(1))\\
    \end{align*}
    由$o(1)$的性质可知$\exists c>0,\,s.t.\, \forall h\in[0,c]$对余项$o(1)$有:$|o(1)|<\frac{|t|}{2}$。
    那么$\forall h\in [0,c],\, f(x+h(x-x^*))=g(h)<f(x^*)+\frac{h^2}{2}(t+\frac{|t|}{2})=f(x^*)+\frac{th^2}{4}<f(x^*)$。
    令$D_1=\{x_1|x_1=x^*+h(x-x^*),\,h\in[0,c]\}$.
    所以对任意$x^*$的邻域$D=B^o_\epsilon(x^*)\bigcap X$,$\exists x_2\in D\bigcap D_1$,使得$f(x_2)<f(x)$。
    这与$x^*$是局部最小点矛盾。

    综上所述,如果$x^*$是凸集$X$二阶连续可微函数$f:\mathbb{R}^n \mapsto \mathbb{R}$的局部最小值,并且有$x\in X,\,s.t\, \nabla f(x^*)^T(x-x^*) = 0$。那么
	必有$(x-x^*)^T\nabla^2f(x^*)(x-x^*) \geq 0$。


    \begin{homeworkProblem}
		A farmer annually producing $x_i$ units of a certain crop stores $(1 − u_i) x_i$ units of his production, where $0 \leq u_i \leq 1$, and invests the remaining $u_ix_i$ units, thus increasing the next year's production
		to a level $x_{i+1}$ given by
		\begin{equation*}
			x_{i+1} = x_i + wu_ix_i, i=0,1,\cdots,N-1,
		\end{equation*}
		where $w$ is a given positive scalar. The problem is to find the optimal investment sequence $u_0,\cdots,u_{N-1}$ that maximizes the total product stored over $N$ years
		\begin{equation*}
			x_N + \sum_{i=0}^{N-1} (1-u_i)x_i.
		\end{equation*}
		Show that one optimal sequence is given by: \\
		(1) If $w>1$, $u_0^*=\cdots=u_{N-1}^*=1$.\\
		(2) If $0<w<\frac{1}{N}$, $u_0^*=\cdots=u_{N-1}^*=0$.\\
		(3) If $\frac{1}{N}\leq w\leq 1$,
		\begin{equation*}
		\begin{aligned}
			& u_0^*=\cdots=u_{N-\bar{i}-1}^*=1,\\
			& u_{N-\bar{i}}^*=\cdots=u_{N-1}^*=0,
		\end{aligned}
		\end{equation*}
		where $\bar{i}$ is the integer such that $1/(\bar{i}+1)<w\leq1/\bar{i}$.
	\end{homeworkProblem}
    \subsection{Solution}
    记$P_n=\prod_{i=0}^{n}(1+wu_i)>0,\,P_{-1}=1$。由题意总的收益函数$Q_N(\vec{u})=x_0((\sum_{i=0}^{N-1} (1-u_i)P_{i-1})+P_{N-1})$
    要使得$Q_N(\vec{u})$最大,即要使得$Q^*_N(\vec{u})=P_{N-1}+\sum_{i=0}^{N-1} (1-u_i)P_{i-1}$最大。

    现在来看$Q^*_N(\vec{u})$关于$u_{N-1}$的项:$(1-u_{N-1})P_{N-2}+(1+wu_{N-1})P_{N-2}=(2+(w-1)u_{N-1})P_{N-2}$。
    由于其它项与$u_{N-1}$无关,所以$Q^*_N(\vec{u})$要取得最大值,必须要这两项取得最大值。而这两项是关于$u_{N-1}$的线性函数
    可以知道当系数大于0时,$u_{N-1}$越大越好(这里最多取1);反之,$u_{N-1}$越小越好(这里最小取0)。
    由于系数与前面的$u_i$无关,所以无论前面哪个$u_i$取什么值,$u_{N-1}$按照这样的取法都可以使收益函数取到最大值。即有$u_{N-1}$的最优反应:
    $\begin{cases}
        u_{N-1}^*=0,\quad w\leq 1\\
        u_{N-1}^*=1,\quad w>1
    \end{cases}$

    像这样从后往前分析,下面我们来证明两个命题。

    \begin{lemma}
        当对$u_{N-1},\,\cdots,u_{N-k-1}$的最优反应均为0时,$u_{N-k}^*$的最优反应为0当且仅当$w\leq \frac{1}{k}$。
    \end{lemma}

    \begin{proof}
        $u_{N-k}$代表着第$N-k$年的投资比例,所以其仅与后面$k+1$年的收益有关。而给定后面$k-1$个最优反应均为0时。
        此时总的收益函数关于$u_{N-k}$的项变为
        \begin{align*}
            f(u_{N-k})&=P_{N-1}+\sum_{i=N-k}^{N-1} (1-u_i)P_{i-1}\\
            &=P_{N-k}+(k-1)P_{N-k}+(1-u_{N-k})P_{N-K-1}\\
            &=(k(1+wu_{N-k})+(1-u_{N-k}))P_{N-k-1}\\
            &=(k+1+(kw-1)u_{N-k})P_{N-k-1}
        \end{align*}

        跟前面关于$u_{N-1}$的分析一样,当系数大于0时,$u_{N-k}$越大越好;反之,$u_{N-k}$越小越好。
        并且系数与前面的$u_0,\cdots,u_{N-k-1}$无关,所以无论前面哪个$u_i$取什么值,$u_{N-k}$按照这样的取法都可以使收益函数取到最大值。
        所以有$u_{N-k}$的最优反应:
        $\begin{cases}
            u_{N-k}^*=0,\quad kw-1\leq0\Leftrightarrow w\leq\frac{1}{k}\\
            u_{N-k}^*=1,\quad kw-1>0\Leftrightarrow w>\frac{1}{k}
        \end{cases}$
    \end{proof}

    \begin{lemma}
        如果在从后往前分析时$u_{N-k+1}$的最优反应分析出来为1,那么$u_{N-k}$的最优反应为1。
    \end{lemma}

    \begin{proof}
        由上此时必有$w>\frac{1}{k}$。此时关于$u_{N-k}$的项$f(u_{N-k})$大于$u_{N-k},\,\cdots,u_{N-1}$全部取0时后面项的和。
        即$f(u_{N-k})$关于$P_{N-k-1}$前面的系数$h$大于$k+1$。那么此时对于总的收益函数中关于$u_{N-k-1}$的项有:
        \begin{align*}
            f(u_{N-k-1})&=h(1+wu_{N-k-1})P_{N-k-2}+(1-u_{N-k-1})P_{N-k-2}\\
            &=(h+1+(hw-1)u_{N-k-1})P_{N-k-2}
        \end{align*}
        此时关于$u_{N-k-1}$的系数$hw-1>(k+1)w-1>(k+1)\frac{1}{k}-1>0$,所以$u_{N-k-1}$的最优反应为1。
    \end{proof}

    综上所述我们对(1),(2),(3)的最优反应有了如下的结论:
    \begin{enumerate}
        \item 当$w>1$时,此时首先对$u_{N-1}$的最优反应为1,而由Lemma2可推知前面所有$u_i$的最优反应均为1,所以$u_0^*=\cdots=u_{N-1}^*=1$。
        \item 当$0<w<\frac{1}{N}$时,此时首先对$u_{N-1}$的最优反应为0,而由Lemma1可推知前面所有$u_i$的最优反应均为0,所以$u_0^*=\cdots=u_{N-1}^*=0$。
        \item 当$\frac{1}{N}\leq w\leq 1$时,此时首先对$u_{N-1}$的最优反应为0,而由Lemma1可推知直到$u_{N-\bar{i}}$的最优反应为0,到$u_{N-\bar{i}-1}$时
        的最优反应变为1,再由Lemma2可推知前面所有$u_i$的最优反应均为1,所以$u_0^*=\cdots=u_{N-\bar{i}-1}^*=1$,而$u_{N-\bar{i}}^*=\cdots=u_{N-1}^*=0$。
    \end{enumerate}

    tips: 其实这题仔细分析后(类似于Lemma2的分析)可知最优的反应一定是前面每个$u_i$取1,而后面所有$u_i$取0。那么假设在$k$截断
    此时收益函数变为:$x_0\sum_{i=k}^{N}\prod_{j=0}^{k-1}(1+wu_j)=x_0(N-k+1)(1+w)^k$,对$k=s,s+1$的收益函数做商即可推出最优的截断点为$N-\bar{i}$。

	\begin{homeworkProblem}
		Consider the problem
		\begin{equation*}
		\begin{aligned}
		& \min_x \frac{f(x)}{g(x)} \\
		& \text{s.t.} x\in X,
		\end{aligned}
		\end{equation*}
		where $f:\mathbb{R}^n\mapsto\mathbb{R}$ and $g:\mathbb{R}^n\mapsto\mathbb{R}$ are given functions and $X$ is a given subset such that $g(x)>0$ for all $x\in X$. For $\lambda\in\mathbb{R}$, define
		\begin{equation*}
		Q(\lambda) = \inf_{x\in X} \{f(x)-\lambda g(x)\},
		\end{equation*}
		and suppose that a scalar $\lambda^*$ and a vector $x^*\in X$ satisfy $Q(\lambda^*)=0$ and 
		\begin{equation*}
		x^*\in\argmin_{x\in X} \{f(x)-\lambda g(x)\}.
		\end{equation*}
		Show that $x^*$ is an optimal solution of the original problem. Use this observation to suggest a solution method that does not require dealing with fractions of functions.
	\end{homeworkProblem}

    \subsection{Solution}
    若存在这样的$\lambda^*$,则有$Q(\lambda^*)=\inf_{x\in X} \{f(x)-\lambda^* g(x)\}=0$,也即$f(x)\geq\lambda^*g(x),\, \forall x\in X$。
    而$g(x)>0\Rightarrow \frac{f(x)}{g(x)}\geq \lambda^*,\,\forall x\in X$。而$x^*\in X ,\,s.t.\, f(x)-\lambda^* g(x)$能取到最小值0,
    即$\frac{f(x^*)}{g(x^*)}=\lambda^*$。所以$\forall x\in X,\,\frac{f(x)}{g(x)}\geq \frac{f(x^*)}{g(x^*)} \Rightarrow x^*$是最优解。

    现在考虑如何将其抽离为算法。首先,引进$\lambda$作为新变量。
    对于每一个$\lambda=\frac{f(x_1)}{g(x_1)}$,我们可以求出$f(x)-\lambda g(x)$在$X$上的最优点,设之为$x_i$
    如果$f(x_i)-\lambda g(x_i)=0$,则由上面的分析说明$x_i$已经是最优点。
    否则一定有$f(x_i)-\lambda g(x_i)<0$,因为$f(x_1)-\lambda g(x_1)=0$,而最优点小于等于之。
    这个时候我们可以用 $\lambda^{'}=\frac{f(x_i)}{g(x_i)}<\lambda$代替之作为新的迭代变量。
    重复这个过程,直到收敛。写成\hyperref[alg1]{算法}:
    \begin{algorithm}[!htb]
        \caption{Fractional programming(Dinkelbach)}
        \begin{algorithmic}[1]\label{alg1}
            \State Input: $x_0\in X,\epsilon>0$;
            \State Initialize $\lambda_0=\frac{f(x_0)}{g(x_0)},\,Q_0=-\infty,\,i=0$;
            \While{$Q_i<-\epsilon$}
                \State Solve $x_{i+1}=\argmin_{x\in X} \{f(x)-\lambda_i g(x)\}$;
                \State Compute $Q_{i+1}=f(x_{i+1})-\lambda_i g(x_{i+1})$;
                \State Update $\lambda_{i+1}=\frac{f(x_{i+1})}{g(x_{i+1})},\,i=i+1$;
            \EndWhile
            \State \Return $x_{i},\lambda_{i}$;
        \end{algorithmic}
    \end{algorithm}

    关于这个算法的收敛性,有一种说法是如果 $f(x),\,g(x)$ 连续且$X$是紧集,或者$f(x),g(x)$凸,且$g(x)>0$则该算法收敛到全局最优解。
    但是我还没验证过。
\end{document}