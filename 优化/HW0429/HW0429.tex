\documentclass[11pt]{article}
\usepackage{listings}
\usepackage{amsmath,amssymb,amsthm}
\usepackage{hyperref}
\usepackage{algpseudocode}
\usepackage{extramarks}
\usepackage{ctex}
\usepackage{tikz}
\usetikzlibrary{shapes.geometric, arrows}

\oddsidemargin 0pt
\evensidemargin 0pt
\marginparwidth 10pt
\marginparsep 10pt
\topmargin -20pt
\textwidth 6.5in
\textheight 8.5in
\parindent = 20pt

\tikzstyle{startstop} = [rectangle, rounded corners, minimum width=3cm, minimum height=1cm,text centered, draw=black]
\tikzstyle{process} = [rectangle, minimum width=3cm, minimum height=1cm, text centered, draw=black]
\tikzstyle{decision} = [diamond, aspect=2, minimum width=3cm, minimum height=1cm, text width=4cm, align=center, draw=black]
\tikzstyle{arrow} = [thick,->,>=stealth]

\DeclareMathOperator*{\argmin}{argmin}
\DeclareMathOperator*{\minimax}{minimax}

\newtheorem{theorem}{Theorem}
\newtheorem{lemma}{Lemma}


\newcommand{\enterProblemHeader}[1]{
	\nobreak\extramarks{}{Problem \arabic{#1} continued on next page\ldots}\nobreak{}
	\nobreak\extramarks{Problem \arabic{#1} (continued)}{Problem \arabic{#1} continued on next page\ldots}\nobreak{}
}

\newcommand{\exitProblemHeader}[1]{
	\nobreak\extramarks{Problem \arabic{#1} (continued)}{Problem \arabic{#1} continued on next page\ldots}\nobreak{}
	\stepcounter{#1}
	\nobreak\extramarks{Problem \arabic{#1}}{}\nobreak{}
}

\setcounter{secnumdepth}{0}
\newcounter{partCounter}
\newcounter{homeworkProblemCounter}
\setcounter{homeworkProblemCounter}{1}
\nobreak\extramarks{Problem \arabic{homeworkProblemCounter}}{}\nobreak{}

\newenvironment{homeworkProblem}[1][-1]{
	\ifnum#1>0
	\setcounter{homeworkProblemCounter}{#1}
	\fi
	\section{Problem \arabic{homeworkProblemCounter}}
	\setcounter{partCounter}{1}
	\enterProblemHeader{homeworkProblemCounter}
}{
	\exitProblemHeader{homeworkProblemCounter}
}

\begin{document}
	
	\title{\bf Homework 6}
	\author{ 陈远洋 \quad 23307130322 }
	\date{\today}
	\maketitle	
	
	\begin{homeworkProblem}
		In the BFGS method, the approximated Hessian is updated by
		$$
		B_{k+1}=B_k-\frac{B_k s_k s_k^T B_k}{s_k^T B_k s_k}+\frac{y_k y_k^T}{y_k^T s_k} ,
		$$
		where $s_k = x_{k+1}-x_k, y_k=\nabla f(x_{k+1})-\nabla f(x_k)$. Show that if $B_k$ is symmetric positive definite, then $B_{k+1}$ is also symmetric positive definite.
				
		Another form of the BFGS update is 
		$$
		H_{k+1}=\left(I-\rho_k s_k y_k^T\right) H_k\left(I-\rho_k y_k s_k^T\right)+\rho_k s_k s_k^T,
		$$
		where $H_k=B_k^{-1}$.
		Apply the Sherman–Morrison–Woodbury formula to verify the second update rule.
	\end{homeworkProblem}

    \textbf{证明:}

        先证一个引理:
        \begin{lemma}
            设$\langle a,b\rangle _A=a^TAb\,(A \succ 0)$表示一个内积,那么这个内积满足Cauchy-Schwarz不等式:
            $$
            \langle a,b\rangle _A^2 \leq \langle a,a\rangle _A\langle b,b\rangle _A 
            $$
        \end{lemma}
        \begin{proof}
        设$t\in \mathbb{R}$,则有$\langle a+tb,a+tb\rangle _A=\langle a,a\rangle _A+2t\langle a,b\rangle _A+t^2\langle b,b\rangle _A\geq 0$
        当且仅当$a,b$同向时左端项关于$t$的最小值能使等号成立。那么这个关于$t$的二次函数极小值大于等于0,即$\langle a,b\rangle _A^2 \leq \langle a,a\rangle _A\langle b,b\rangle _A $
        引理得证。
        \end{proof}

        回到原题。首先,显然有:$B_{k+1}^T=B_{k+1}$。故$B_{k+1}$对称。由条件$\nabla f(x_{k+1})^Td_k\geq \nabla f(x_k)^Td_k$可知$y_k^Ts_k>0$。对任意$x\in\mathbb{R}^n$,且x与$x_k$不共线时,
        由Cauchy-Schwarz不等式,有:
        \begin{align*}
            x^TB_{k+1}x&=\langle x,x\rangle _{B_{k}}-\dfrac{\langle x,s_k\rangle _{B_{k}}^2}{\langle s_k,s_k\rangle _{B_{k}}}+\dfrac{(x^Ty_k)^2}{y_k^Ts_k}\\
            &> \langle x,x\rangle _{B_{k}} - \dfrac{\langle x,x\rangle _{B_{k}}\langle s_k,s_k\rangle _{B_{k}}}{\langle s_k,s_k\rangle _{B_{k}}}+\dfrac{(x^Ty_k)^2}{y_k^Ts_k}\\
            &=\dfrac{(x^Ty_k)^2}{y_k^Ts_k}\\
            &\geq0
        \end{align*}
        在$x$与$s_k$同向时第二步等号成立,但第4步不等号严格成立,故在$s_k$与$x$共线时,也有$x^TB_{k+1}x>0$。
    
        综上,$B_{k+1}$是对称正定矩阵。

        接下来证明$H_{k+1}$更新表达式的正确性:令$U=\begin{pmatrix}y_k&B_ks_k\end{pmatrix}$,$A=\begin{pmatrix}\dfrac{1}{y_k^Ts_k}&0\\0&-\dfrac{1}{s_k^TB_ks_k}\end{pmatrix}$。
        有:$B_{k+1}=B_k+UAU^T$。由Shellman-Morrison-Woodbury公式,有:
        \begin{align*}
            H_{k+1}&=H_k-H_kU(A^{-1}+U^TH_kU)^{-1}U^TH_k\\
            &=H_k-\begin{pmatrix}H_ky_k&s_k\end{pmatrix}(\begin{pmatrix}y_k^Ts_k&0\\0&-s_k^TB_ks_k\end{pmatrix}+\begin{pmatrix}y_k^TH_ky_k&y_k^TH_kB_ks_k\\s_k^TB_kH_ky_k&s_k^TB_kH_kB_ks_k\end{pmatrix})^{-1}\begin{pmatrix}y_k^TH_k\\s_k^T\end{pmatrix}\\
            &=H_k-\begin{pmatrix}H_ky_k&s_k\end{pmatrix}\begin{pmatrix}y_k^Ts_k+y_k^TH_ky_k&y_k^Ts_k\\s_k^Ty_k&0\end{pmatrix}^{-1}\begin{pmatrix}y_k^TH_k\\s_k^T\end{pmatrix}\\
            &=H_k+\dfrac{1}{(y_k^Ts_k)^2}\begin{pmatrix}H_ky_k&s_k\end{pmatrix}\begin{pmatrix}0&-y_k^Ts_k\\-s_k^Ty_k&y_k^Ts_k+y_k^TH_ky_k\end{pmatrix}\begin{pmatrix}y_k^TH_k\\s_k^T\end{pmatrix}\\
            &=H_k+\dfrac{1}{(y_k^Ts_k)^2}\begin{pmatrix}H_ky_k&s_k\end{pmatrix}\begin{pmatrix}-y_k^Ts_ks_k^T\\y_k^TH_ky_ks_k^T+y_k^Ts_ks_k^T-s_k^Ty_ky_k^TH_k\end{pmatrix}\\
            &=H_k+\dfrac{1}{(y_k^Ts_k)^2}\begin{pmatrix}-H_ky_ky_k^Ts_ks_k^T+s_ky_k^TH_ky_ks_k^T-s_ks_k^Ty_ky_k^TH_k+s_ky_k^Ts_ks_k^T\end{pmatrix}\\
            &=H_k+\dfrac{s_ky_k^TH_ky_ks_k^T}{(y_k^Ts_k)^2}-\dfrac{H_ky_ks_k^T}{y_k^Ts_k}-\dfrac{s_ky_k^TH_k}{y_k^Ts_k}+\dfrac{s_ks_k^T}{y_k^Ts_k}\\
            &=\left(I-\rho_k s_k y_k^T\right) H_k\left(I-\rho_k y_k s_k^T\right)+\rho_k s_k s_k^T
        \end{align*}
        其中$\rho_k=\dfrac{1}{y_k^Ts_k}$.

        综上,$H_{k+1}$更新表达式的正确性证毕。

	\begin{homeworkProblem}
		There are many other quasi-Newton updating formulae apart from BFGS, DFP, and SR1 method. Of particular interest is the Broyden class, a family of updates specified by the following general formula:
		\begin{equation}
			\label{broyden}
			B_{k+1}=B_k-\frac{B_k s_k s_k^T B_k}{s_k^T B_k s_k}+\frac{y_k y_k^T}{y_k^T s_k}+\phi_k\left(s_k^T B_k s_k\right) v_k v_k^T,
		\end{equation}
		where $\phi_k$ is a scalar and $v_k$ is defined as
		$$
		v_k=\left[\frac{y_k}{y_k^T s_k}-\frac{B_k s_k}{s_k^T B_k s_k}\right].
		$$
		Show that the formula \eqref{broyden} is a ``linear combination'' of DFP and BFGS method as
		$$
		B_{k+1}=\left(1-\phi_k\right) B_{k+1}^{\text {BFGS}}+\phi_k B_{k+1}^{\text {DFP}},
		$$
		where $B_{k+1}^{\text {BFGS}}$ and $B_{k+1}^{\text {DFP}}$ denotes $B_{k+1}$ in BFGS and DFP method. Also verify the secant equation holds for $B_{k+1}$ in \eqref{broyden}.
	\end{homeworkProblem}

    \textbf{证明:}

    首先,我们知道,BFGS方法的更新公式为:
    \begin{equation}
        B_{k+1}^{\text {BFGS}}=B_k-\frac{B_k s_k s_k^T B_k}{s_k^T B_k s_k}+\frac{y_k y_k^T}{y_k^T s_k}       
    \end{equation}
    而DFP方法的更新公式为:
    \begin{equation}\label{dfp}
        H_{k+1}^{\text{DFP}}=H_k-\frac{H_ky_ky_k^TH_k}{y_k^TH_ky_k}+\frac{s_ks_k^T}{s_ky_k^T}        
    \end{equation}
    由Problem1的证明可知,我们可以进一步将\hyperref[dfp]{DFP方法的更新公式}写成关于$B_k$的表达式:
    \begin{equation}\label{dfp2}
        B_{k+1}^{\text {DFP}}=B_k+\dfrac{y_ky_k^T}{s_k^Ty_k}+\dfrac{s_k^TB_ks_k}{(s_k^Ty_k)^2}y_ky_k^T-\dfrac{B_ks_ky_k^T+y_ks_k^TB_k}{s_k^Ty_k}
    \end{equation}

    而对$\phi_k$引导的尾项,我们有:
    \begin{align*}
        (s_k^TB_ks_k)v_kv_k^T&=(s_k^TB_ks_k)(\frac{y_k}{y_k^Ts_k}-\frac{B_ks_k}{s_k^TB_ks_k})(\frac{y_k^T}{s_k^Ty_k}-\frac{s_k^TB_k}{s_k^TB_ks_k})\\
        &=\dfrac{s_k^TB_ks_k}{(s_k^Ty_k)^2}y_ky_k^T-\dfrac{B_ks_ky_k^T+y_ks_k^TB_k}{s_k^Ty_k}+\dfrac{B_ks_ks_k^TB_k}{s_k^TB_ks_k}\\
    \end{align*}

    所以,我们可以将\hyperref[dfp2]{关于$B_{k+1}$的DFP方法的更新公式}可以写成:
    \begin{equation}
        B_{k+1}^{\text{DFP}}=B_{k+1}^{\text {BFGS}}+\left(s_k^T B_k s_k\right) v_k v_k^T
    \end{equation}

    那么\hyperref[broyden]{(1)式}可改写为:
    \begin{align*}
        B_{k+1}&=B_{k+1}^{\text{BFGS}}+\phi_k\left(s_k^T B_k s_k\right) v_k v_k^T\\
        &=B_{k+1}^{\text{BFGS}}+\phi_k(B_{k+1}^{\text {DFP}}-B_{k+1}^{\text {BFGS}})\\
        &=(1-\phi_k)B_{k+1}^{\text{BFGS}}+\phi_k B_{k+1}^{\text{DFP}}
    \end{align*}

    接下来验证所谓的secant equation。直观上,我们有$H_{k+1}^{\text{DFP}}y_k=s_k,B_{k+1}^{\text{BFGS}}s_k=y_k$,所以有:
    $B_{k+1}^{\text{DFP}}s_k=y_k$。可以推出$((1-\phi_k)B_{k+1}^{\text{BFGS}}+\phi_k B_{k+1}^{\text{DFP}})s_k=(1-\phi_k)y_k+\phi_ky_k=y_k$。

    而更进一步地,我们也只需验证$v_k^Ts_k=0$即可。而:
    \begin{align*}
        v_k^Ts_k&=\left(\frac{y_k}{y_k^T s_k}-\frac{B_ks_k}{s_k^TB_ks_k}\right)^Ts_k\\
        &=\dfrac{y_k^Ts_k}{y_k^T s_k}-\dfrac{s_k^TB_ks_k}{s_k^TB_ks_k}\\
        &=1-1\\
        &=0
    \end{align*}

    所以\hyperref[broyden]{(1)式}引导的更新表达式满足所谓的secant equation。

	\begin{homeworkProblem}		
		(\textbf{Coding problem}) Consider the logistic regression problem for binary classification: 
		$$
		\min_x \frac{1}{m}\sum_{i=1}^{m}\log\left(1+\exp(-b_ia_i^Tx)\right) + \mu\|x\|_2^2,
		$$
		where $x\in\mathbb{R}^n, a_i\in\mathbb{R}^n$, $b_i\in\{\pm 1\}$ denotes the class that $a_i$ belongs to, $n$ is the dimension of the variable, $m$ is the number of data. $\mu$ is the regularization parameter. 
		
		Please use a quasi-Newton method to solve above problem, using the \textbf{a1a} data given in LIBSVM datasets, or data generated by yourself. 
		A detailed description of the data can be found in \url{https://www.csie.ntu.edu.tw/~cjlin/libsvmtools/datasets/binary.html#a1a}. Please write a report to illustrate the method you used and present your experiment results.\\
		
		Note: 
		\begin{itemize}
			\item The gradient of the objective function is $$
			\nabla f(x)=\frac{1}{m} \sum_{i=1}^m \frac{1}{1+\exp \left(-b_i a_i^T x\right)} \cdot \exp \left(-b_i a_i^T x\right) \cdot\left(-b_i a_i\right)+2 \mu x
			$$
			\item The regularization parameter $\mu$ can be set as $\mu=1e-3$.
			\item If you use `a1a' dataset, you only need to use the `a1a' file, where $m=1605$, and $n=123$. In the file, the first column of each row denotes the label ($b_i$) of each data, and the following rows indicate elements of which coordinates of $a_i$ are non-zero.
			\item If you generate the data by yourself, you can set $m\geq100, n\geq50$.
		\end{itemize}
		
	\end{homeworkProblem}
	\subsection{解存在性分析}
    首先,该问题一定是有解并且唯一的。前面的求和部分与后面正则项在一定范围内是有界的,并且前面部分是恒正的。不妨设有0点附
    近的领域$D=\{x|\mu\|x\|_2^2\leq f(0)\}$。由weirstrass定理,我们知道在$D$内存在原函数最小值$f(x_0)\leq f(0)$。
    而在$D$之外,$f(x)>\mu\|x\|_2^2>f(0)\geq f(x_0)$。所以$x_0$也是全局最小值。另外,正则化项$\mu \|x\|_2^2$是严格凸函数。
    而单看前面的求和中的每一项,对于函数$h(z)=\log(1+\exp(-z))$,其导数为$h'(z)=\frac{-exp(-z)}{1+\exp(-z)}$,
    而$h''(z)=\frac{exp(-z)}{(1+\exp(-z))^2}>0$。所以这个函数也是严格凸的。所以$x_0$也是全局唯一的最小值。

    \subsection{代码构成分析}
    求解这个问题的梯度方法以及拟牛顿方法的通用步骤:

    \begin{tikzpicture}[node distance=2cm, auto]
        \node (start) [startstop] {读取数据$A$、$b$,设置参数$\mu$、初值、精度要求、最大迭代次数};
        \node (pro1) [process, below of=start] {选择方向$d$、步长$\alpha$};
        % 这里的文本会根据设定的 text width 自动换行,但菱形的大小是固定的
        \node (dec1) [decision, below of=pro1, yshift=-1cm] {判断是否有梯度精度达到要求\\或者迭代次数达到最值};
        \node (pro2a) [process, below of=dec1, yshift=-1cm] {结束迭代,输出结果};
        \node (pro2b) [process, right of=dec1, xshift=5.5cm] {更新选择方向、步长所需参数};
        \draw [arrow] (start) -- (pro1);
        \draw [arrow] (pro1) -- (dec1);
        \draw [arrow] (dec1) -- node[anchor=east] {是} (pro2a);
        \draw [arrow] (dec1) -- node[anchor=south] {否} (pro2b);
        \draw [arrow] (pro2b) |- (pro1);    
    \end{tikzpicture}

    在主函数中我主要实现初始化数据阶段,以及效果比较。对拟牛顿法的实现上,我写了BFGS和DFP两个方法。而迭代的过程则可以通过调用不同接口而对应不同方法。
    下面来看各个辅助函数以及其对应的作用:

    \begin{itemize}
        \item \texttt{logisticRegression}: 对应优化的目标函数,输入参数为$\mu$、$A$、$b$、$x$。
        在后续使用过程中,可以通过在句柄中设定参数将其变为关于x的单变量函数。
        \item \texttt{grad\_LR}: 顾名思义,上一个函数关于x的梯度。
        \item \texttt{LineSearch}: 输入参数$f$、$df$、$d$ 、$x$ 、$a$、$c$ 、$\beta$。输出在$d$方向下满足$armijo$条件
        的以$a$为初始值,每次递减$\beta$的最大步长。但是实际实现中,为了确保拟牛顿算法$H$的正定性,我们要求$\nabla f(x+ad)^Td<\nabla f(x)^Td$
        (相当于加了一个c2=1的wolfe准则)。但是值得注意的是,这个条件非常宽松,以至于在本次实验中我没有遇到满足前一个条件而不满足这个条件
        的情况,所以LineSearch的过程可以认为是$armijo$的。我设定$a=1.5$,$c=0.01$, $\beta=0.99$。
        在后续使用中,可以将已经调试好的参数通过固定在句柄中的方式调用。
        \item \texttt{BFGS}: 对应BFGS的解决方法。输入参数$f$、$df$、$x_0$、$maxIter$、$tol$。输出$x$的BFGS迭代值。
        值得说明的是我们知道BFGS关于$H$的迭代表达式:
        \[H_{k+1}=(I-\rho_k s_k y_k^T)H_k(I-\rho_k y_k s_k^T)+\rho_k s_ks_k^T=H_k-(t+t^T)+(\rho_k+\rho_k^2y_k^TH_ky_k)s_ks_k^T\]
        其中$\rho_k=\frac{1}{s_k^Ty_k},t=\rho_kH_ky_ks_k^T$,通过预先计算$t$跟$\rho_k$,我们可以避免矩阵与矩阵之间的乘法,
        把单次更新的时间复杂度降为$O(n^2)$。
        \item \texttt{DFP}: 对应DFP的解决方法。输入参数$f$、$df$、$x_0$、$maxIter$、$tol$。输出$x$的DFP迭代值。
        DFP的迭代过程与BFGS类似,总之注意加上括号改变运算顺序,避免矩阵与矩阵之间的直接乘法。保证单次更新时间复杂度在$O(n^2)$级别。
    \end{itemize}
\subsection{参数说明}
    关于为什么步长初值取成1.5,是出于选择大步长以及减小每次迭代的抖动的综合考量。
    以DFP为例,当步长初值较小时,可能每次向前“迈的步子”太小,导致离当前方向下的最优解很远。
    而这也会导致梯度模长的“抖动”。当步长初值较大时,可能每次向前“迈的步子”太大,导致迭代过程过于不稳定。
    出于这样的综合考量,在实验中,我试出来以1.5为初值步长的效果最好。
    详见\hyperref[fig:ab]{下图}:
    \begin{figure}[!htb]\label{fig:ab}
        \centering
        \includegraphics[width=0.45\textwidth]{picture/a.jpg}\hspace{1em}
        \includegraphics[width=0.45\textwidth]{picture/b.jpg}
        \caption{初始步长过大或者过小导致的抖动(左图初始步长为1,右图初始步长为2)}
    \end{figure}
\subsection{实验结果}
    最优解采取BFGS得到的结果,放在x1.mat文件中。当设置停机条件为梯度模长小于$1\times 10^{-8}$时,最优解对应的函数值为$0.33691514$。
    达到终点时,BFGS迭代次数为144,DFP迭代次数为513。二者终点处达到的梯度模长分别
    为$8.95\times10^{-9}$跟$7.35\times10^{-9}$。

    \begin{figure}[!htb]
        \centering
        \includegraphics[width=0.45\textwidth]{picture/res.png}
        \caption{最终结果(加上SD)预期输出}
    \end{figure}

    每步迭代后函数值减去最小值取对数的图像以及梯度模长变化如图\hyperref[f3]{3}、\hyperref[f4]{4}、\hyperref[f5]{5}所示:
    \begin{figure}[!htb]\label{f3}
        \centering
        \includegraphics[width=0.45\textwidth]{picture/1-1.jpg}\hspace{1em}
        \includegraphics[width=0.45\textwidth]{picture/1-2.jpg}
        \caption{BFGS的收敛情况}
    \end{figure}
    \begin{figure}[!htb]\label{f4}
        \centering
        \includegraphics[width=0.45\textwidth]{picture/2-1.jpg}\hspace{1em}
        \includegraphics[width=0.45\textwidth]{picture/2-2.jpg}
        \caption{DFP的收敛情况}
    \end{figure}
    \begin{figure}[!htb]\label{f5}
        \centering
        \includegraphics[width=0.45\textwidth]{picture/3-1.jpg}\hspace{1em}
        \includegraphics[width=0.45\textwidth]{picture/3-2.jpg}
        \caption{最速下降法(SD)的收敛情况}
    \end{figure}

    可以看到总体上大致是一个线性的过程,整体上呈现微弱的超线性收敛趋势,这似乎与我们之前的预期不太相符。
    但为了验证拟牛顿法的优越性,我们加入最速下降的梯度法作为对比。达到相同精度时,最速下降法的迭代次数为3761。
    所以在这个问题上我们的拟牛顿方法远远快于梯度方法。
\end{document}