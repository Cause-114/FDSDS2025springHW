\documentclass[UTF8]{ctexart}
\usepackage{amsmath, amsthm, amssymb, graphicx,caption,capt-of}
\usepackage[a4paper, total={6in, 9in}]{geometry}
\usepackage[colorlinks=true]{hyperref}
\title{2024$\sim$2025春数据库及实现上机实验报告一}
\author{陈远洋}
\date{\today}
\begin{document}
\maketitle
\section{实验环境说明}
    我的本地操作系统为Windows11,64 位操作系统, 基于 x64 的处理器。
    安装的MySQL版本信息: \verb|Ver 8.0.41 for Win64 on x86_64 (MySQL Community Server - GPL)|。
    实验中使用的数据库软件为MySQL Workbench 8.0.41。

\section{任务1:概念模型(E-R图)画法与逻辑模式转换实验}
\subsection{实验目的}
\begin{enumerate}
    \item 掌握E-R图的构成要素及图元。
    \item 学习如何绘制图书馆管理系统的E-R图。
    \item 掌握从概念模型(E-R图)向逻辑模型(关系模式)的转换原则和步骤。
\end{enumerate}
\subsection{实验内容}
    题目1:为一个图书馆设计一套图书管理的数据库,要求包括出版社和书籍的信息。
    出版社信息包括出版社名称、地址、电话;书籍信息包括书名、作者、ISBN、价格、出版日期。
    \begin{enumerate}
        \item 确定书籍实体和出版社实体的属性。(2分)\\
        书籍(Book):书名BN, 作者AT, 国际标准书号ISBN, 价格P。\\
        出版社(Press):出版社名称PN, 地址AD, 电话TEL。
        \item 确定书籍和出版社之间的联系,给联系命名并指出联系的类型。(2分)\\
        联系名称:出版\\
        联系类型:一对多(出版社对应书籍)
        \item 确定联系本身的属性。(2分)\\
        出版属性:出版日期DATE
        \item 画出书籍与出版社关系的E-R图。(4分)\\
        见\hyperref[fig:1,2]{图一}。
        \item 将E-R图转化为关系模式,写出表的关系模式并标明各自的码。(2分)\\
        实体“书籍”$\rightarrow$  Book(\underline{ISBN}, BN, AT, P, DATE, PN) PK: ISBN, FK: PN\\
        实体“出版社”$\rightarrow$ Press(\underline{PN}, AD, TEL) PK: PN
    \end{enumerate}
    题目2:设计一个图书馆员工和会员的管理系统。图书馆员工信息包括员工编号
    、姓名、电话等;会员信息包括会员编号、姓名、电话等。员工与会员之间存
    在“服务”联系,每个员工可服务多个会员,但每个会员只能由一个员工服务,
    员工服务会员有“开始服务日期”和“服务次数”两个属性;员工与书籍之间存在
    “管理”联系,每个员工可管理多本书籍,而每本书籍只能由一个员工管理;会
    员与书籍之间存在“借阅”联系,会员借阅书籍有“借阅日期”和“归还日期”两个
    属性,每个会员可借阅多本书籍,每本书籍可被多个会员借阅。   
    \begin{enumerate}
        \item 确定实体和实体的属性。(2分)\\
        员工:员工编号SID, 姓名SN, 电话STEL\\
        会员:会员编号VID, 姓名VN, 电话VTEL
        \item 确定实体之间的联系,给联系命名并指出联系的类型。(2分)\\
        员工$\leftrightarrow$会员,命名:服务 类型:1对多联系\\
        员工$\leftrightarrow$书籍,命名:管理 类型:1对多联系\\
        会员$\leftrightarrow$书籍,命名:借阅 类型:多对多联系
        \item 确定联系本身的属性。(2分)\\
        服务:开始服务日期SD,服务次数ST\\
        管理:\\
        借阅:借阅日期BD, 归还日期ED
        \item 画出E-R图。(4分) \\
        见\hyperref[fig:1,2]{图二}。
        \item 将E-R图转化为关系模式,写出表的关系模式并标明各自的码。(3分)\\
        实体“员工”$\rightarrow$ Staff(\underline{SID}, SN, STEL) PK: SID\\
        实体“书籍”$\rightarrow$ Book(\underline{ISBN}, BN, AT, P, DATE, PN, SID) PK: ISBN; FK: PN, SID\\
        实体“会员”$\rightarrow$ VIP(\underline{VID}, VN, VTEL, SID, SD, ST) PK:VID; FK:SID\\
        联系“借阅”$\rightarrow$ Borrow(\underline{VID,ISBN,BD},ED) PK:VID+ISBN+BD; FK:VID,ISBN
    \end{enumerate}
    \begin{figure}[!htb]\label{fig:1,2}
        \includegraphics[width=0.45\textwidth]{newpic/ER1.png}\hfill
        \includegraphics[width=0.45\textwidth]{newpic/ER2.png}\caption{图一、图二E-R图}
    \end{figure}
\section{任务2:关系的完整性、规范化理解与应用实验}
\subsection{实验目的}
\begin{enumerate}
    \item 了解关系模型的基本概念,掌握候选码和主码的确定。
    \item 掌握并应用完整性规则。
    \item 掌握关系规范化的定义和方法。
\end{enumerate}
\subsection{设计性实验}
某医院设计了药品库存管理系统,设计了如下\hyperref[tab:1]{药品订单表},请你用规范化理论将该表进行分解,
使之满足3NF的规范化要求。(10分)
\begin{table}[!htb]
    \centering\small
    \caption{药品订单表}
    \label{tab:1}
    \begin{tabular}{|c|c|c|c|c|c|c|c|c|}
        \hline
        订单号&药品代码&药品名称&生产厂家&厂家地址&单价&订购数量&订购医院&医院地址\\\hline
        OD20190301&PH5002&阿莫西林&华北制药&河北省石家庄&12&100&人民医院&江苏省南京市\\\hline
        OD20190302&PH0038&感冒灵&仁和药业&浙江省杭州市&8&200&同仁医院&浙江省杭州市\\\hline
        OD20190303&PH1073&板蓝根&康恩贝集团&江苏省南通市&15&150&协和医院&北京市\\\hline
        OD20190304&PH2042&肠胃宁&修正药业&辽宁省沈阳市&20&80&同济医院&上海市\\\hline
        OD20190305&PH5002&阿莫西林&华北制药&河北省石家庄&12&120&中日医院&北京市\\
        \hline
    \end{tabular}
\end{table}

    考虑到三范式的定义,即表中没有传递函数依赖。而此表中有药品代码$\rightarrow$药品名称+生产厂家,生产厂家$\rightarrow$厂家地址,
    订购医院$\rightarrow$医院地址,订单号$\rightarrow$药品代码+订购医院+订购数量+单价,等函数依赖可推出的部分函数依赖、传递函数依赖。
    所以需要进行分解。
    
    将原表格分解为三个独立实体:药品,厂家,医院;以及一个联系关系:订单。分解如下:
    药品(\underline{药品代码},药品名称,\textbf{生产厂家},单价),厂家(\underline{厂家名称},厂家地址),
    医院(\underline{医院名称},医院地址),订单(\underline{订单号},\textbf{药品代码}
    ,\textbf{订购医院},订购数量)。(画下划线的属性为主键,加粗的属性为外键)见\hyperref[tab:2]{表2}.
    \begin{table}[!htb]
        \centering\small\label{tab:2}
        \caption{分解后的表格}
        \begin{tabular}{|c|c|c|c|}
            \hline
            \underline{药品代码}&药品名称&生产厂家&单价\\\hline
            PH5002&阿莫西林&华北制药&12\\\hline
            PH0038&感冒灵&仁和药业&8\\\hline
            PH1073&板蓝根&康恩贝集团&15\\\hline
            PH2042&肠胃宁&修正药业&20\\\hline
        \end{tabular}\hspace{2em}
        \begin{tabular}{|c|c|}
            \hline
            \underline{厂家名称}&厂家地址\\\hline
            华北制药&河北省石家庄\\\hline
            仁和药业&浙江省杭州市\\\hline
            康恩贝集团&江苏省南通市\\\hline
            修正药业&辽宁省沈阳市\\\hline
        \end{tabular}\vspace{1em}
        \begin{tabular}{|c|c|c|c|}
            \hline
            \underline{订单号}&药品代码&订购医院&订购数量\\\hline
            OD20190301&PH5002&人民医院&100\\\hline
            OD20190302&PH0038&同仁医院&200\\\hline
            OD20190303&PH1073&协和医院&150\\\hline
            OD20190304&PH2042&同济医院&80\\\hline
            OD20190305&PH5002&中日医院&120\\\hline            
        \end{tabular}\hspace{2em}
        \begin{tabular}{|c|c|}
            \hline
            \underline{医院名称}&医院地址\\\hline
            人民医院&江苏省南京市\\\hline
            同仁医院&浙江省杭州市\\\hline
            协和医院&北京市\\\hline
            同济医院&上海市\\\hline
            中日医院&北京市\\\hline            
        \end{tabular}
    \end{table}
\subsection{观察与思考}
1. 有如下所示的两张表:假设向关系M中插入新行,新行的数据分别如下。哪些行能够插入?若不能插入,为什么?(3分)
\begin{table}[!htb]
    \centering
    \caption{供应商关系S(主码是“供应商ID”)}
    \begin{tabular}{|c|c|c|}\hline
        供应商ID&供应商名称&所在城市\\\hline
        S01&健康药业&南京\\\hline
        S02&万艾可&北京\\\hline
        S03&长生堂&广州\\\hline
        S04&九芝堂&成都\\\hline
    \end{tabular}
\end{table}
\begin{table}[!htb]
    \centering
    \caption{药品关系M(主码是“药品ID”,外码是“供应商ID”)}
    \begin{tabular}{|c|c|c|}\hline
        药品ID&药品名称&供应商ID\\\hline
        P001&阿司匹林&S01\\\hline
        P002&复方感冒胶囊&S02\\\hline
        P003&维生素C&S03\\\hline
    \end{tabular}
\end{table}
\\A. (`P004', `银翘解毒片', `S05')\\
B. (`P005', `感冒灵', null)\\
C. (`P003', `板蓝根', `S03')\\
D. (`P006', `藿香正气水', `S01')\\
E. (`P007', `牛黄解毒片', `S07')\\

A, C, E不能。A, E是因为外键在S中取值不存在;C是因为主键冲突。
B, D可以。因为外键取值为S中的值或者空。

2. 非规范化数据表可能带来哪些不利影响?(3分)

较多的数据冗余,较低的数据结构化程度以及可能带来的数据更新异常
较低的数据共享度以及可能带来的数据一致性较差。

4. 在规范化过程中,为什么要避免传递依赖?(3分)

若存在传递依赖可能导致数据冗余:对一些传递依赖的列可能会被多次重复。
更新,插入,删除异常:在修改某个规则时,可能需要对表中成千上万的键进行修改,如此带来大量的工作量以及可能导致数据一致性被破坏。

5. 如果一个关系已经处于2NF,它还可能存在哪些问题?(3分)

可能仍然存在传递函数依赖,即不满足3NF。此时可能遇到的问题同上即:
\begin{itemize}
    \item 数据冗余
    \item 更新异常
    \item 插入异常
\end{itemize}

\section{任务3: MySQL安装创建和维护数据库实验}
\subsection{实验目的}
\begin{enumerate}
    \item 熟悉在Windows、Mac或Linux平台下安装与配置MySQL的方法。
    \item 掌握启动服务并登录MySQL数据库的多平台方法和步骤。
    \item 了解手工配置MySQL的跨平台方法。
    \item 深入理解MySQL数据库的相关概念。
    \item 掌握使用MySQL Workbench/Navicat等客户端工具和SQL语句在多平台创建数据库的方法。
    \item 掌握使用MySQL Workbench/Navicat等客户端工具和SQL语句在多平台删除数据库的方法。
\end{enumerate}
\subsection{实验内容}
\begin{enumerate}
    \item 在Windows、Mac或Linux平台下安装MySQL。(2分)\\
    从MySQL官网下载MySQL Installer。运行安装程序,选择“Developer Default”选项。根据提示完成安装。
    我这里选择\verb|Ver 8.0.41 for Win64 on x86_64 (MySQL Community Server - GPL)|。
    \item 在服务(Windows)/系统偏好设置(Mac)/系统服务(Linux)中,手动启动或者关闭MySQL服务。(2分)\\
    打开“服务”管理器(Win + R -> 输入services.msc)。找到“MySQL”服务,右键选择“启动”或“停止”。\hyperref[fig:3]{见图3}
    \item 使用命令行在各平台启动或关闭MySQL服务。(2分)\\
    打开终端管理员模式下的powershell,分别输入
    \begin{verbatim}
        net start mysql80
        net stop mysql80
    \end{verbatim}以开关mysql服务。
    \item 分别用MySQL Workbench/Navicat等客户端工具和命令行方式在各平台登录MySQL。(2分)\\
    下载MySQL Workbench按照提示完成安装,新建连接。对命令行模式输入\verb|mysql -u root -p|,输入密码,进入MySQL命令行。
    \hyperref[fig:3]{见图3}
    \item 在配置文件(my.ini或my.cnf)中将数据库的存储位置改为不同的路径,例如 \verb|D:\MYSQL\DATA| 
    或 \verb|/var/mysql/data|,然后重启服务,并验证路径更改的有效性。(2分)\\
    找到配置文件my.ini,修改[mysqld]部分的datadir参数,并将data目录下的文件拷贝到目标地址,重启服务。
    \hyperref[fig:3]{见图3}
    \item 创建数据库。
    \begin{enumerate}
        \item 使用MySQL Workbench/Navicat等客户端工具在各平台创建教学管理数据库JXGL。(2分)\\
        在MySQL Workbench中,点击“创建数据库”。输入名称“JXGL”,选择字符集,点击确定。
        \item 使用SQL语句在各平台创建数据库MyTestDB。(2分)\\
        在SQL语句查询栏中输入\verb|CREATE DATABASE MyTestDB;|,然后点击运行。
        \hyperref[fig:3]{见图3}。
    \end{enumerate}
    \item 查看数据库属性。
    \begin{enumerate}
        \item 在MySQL Workbench/Navicat等客户端工具中查看创建后的JXGL数据库和MyTestDB数据库的状态,及其文件所在的文件夹。(2分)\\
        在MySQL Workbench中,右键数据库“JXGL”或“MyTestDB”,查看其状态和存储路径。或者键入SQL语句
        \begin{verbatim}
        SELECT * from information_schema.SCHEMATA 
        WHERE SCHEMA_NAME IN ('JXGL','MyTestDB');
        \end{verbatim}
        \item 使用SHOW DATABASES命令在各平台显示当前的所有数据库。(2分)\\
        \hyperref[fig:3]{见图3}。
    \end{enumerate}
    \item 删除数据库。
    \begin{enumerate}
        \item 使用MySQL Workbench/Navicat等客户端工具在各平台删除JXGL数据库。(2分)\\
        在MySQL Workbench中,右键数据库“JXGL”,选择“删除”。
        \item 使用SQL语句在各平台删除MyTestDB数据库。(2分)\\
        键入SQL语句\verb|DROP DATABASE MyTestDB|,然后点击运行。
        \item 使用SHOW DATABASES命令在各平台显示当前的所有数据库。(2分)\\
        完整结果\hyperref[fig:3]{见图3}。
    \end{enumerate}
\end{enumerate}
\begin{figure}[!htb]\label{fig:3}
    \centering
    \includegraphics[width=0.4\textwidth]{newpic/3-2.png}\hspace{1em}
    \includegraphics[width=0.4\textwidth]{newpic/3-3.png}\vspace{1em}
    
    \includegraphics[width=0.4\textwidth]{newpic/3-5.png}\hspace{1em}
    \includegraphics[width=0.4\textwidth]{newpic/3-6-1.png}\vspace{1em}

    \includegraphics[width=0.4\textwidth]{newpic/3-6-2.png}\hspace{1em}
    \includegraphics[width=0.4\textwidth]{newpic/3-7-1.png}\vspace{1em}

    \includegraphics[width=0.4\textwidth]{newpic/3-7-2.png}\hspace{1em}
    \includegraphics[width=0.4\textwidth]{newpic/3-8-1.png}\vspace{1em}

    \includegraphics[width=0.4\textwidth]{newpic/3-8-2.png}\hspace{1em}
    \includegraphics[width=0.4\textwidth]{newpic/3-8-3.png}
\end{figure}
\subsection{观察与思考}
(1)如何在不同平台备份MySQL数据库?\\
根据查找到的资料,可以使用`mysqldump'工具来备份MySQL数据库。基本命令格式如下:
\begin{verbatim}
    mysqldump -u [username] -p[password] [database_name] > backup.sql    
\end{verbatim}

(2)讨论跨平台使用MySQL时的字符编码问题,如何配置以支持多语言?\\
可以将MySQL服务器的默认字符集设置为`utf8mb4',其包括emoji在内的所有Unicode字符。

(3)如何在不同平台上通过配置文件定制MySQL的内存使用和连接限制?\\
如果需要定制内存使用和连接限制,可以调整my.ini文件中的以下参数:
\begin{itemize}
    \item 内存使用:\verb|innodb_buffer_pool_size, key_buffer_size, tmp_table_size, max_heap_table_size|
    \item 连接限制:\verb|max_connections|
\end{itemize}

(4)在分布式部署时,MySQL的数据同步有哪些方式?讨论它们的优势和适用场景。
\begin{itemize}
    \item \textbf{主从复制:} 最常用的同步方式之一,适用于读写分离、负载均衡等场景。优点是可以增加读性能,但不能自动解决冲突。
    \item \textbf{半同步复制:} 在主从复制基础上增加了安全性,保证至少一个从库接收到了事务后再确认给客户端,适用于对数据一致性要求较高的场景。
    \item \textbf{组复制(MySQL Group Replication):} 提供了高可用性和自动故障转移功能,适用于需要高可用性的场景,但它对网络延迟敏感。
    \item \textbf{Galera Cluster:} 一种多主同步复制解决方案,适合于需要多主写入的环境,但其性能可能受到节点数量的影响。
\end{itemize}
选择哪种方式取决于具体的业务需求,比如是否需要多主写入、对一致性的要求、以及预期的读写比例等。

(5)MySQL的数据库文件在不同操作系统中存储位置可能有哪些不同?扩展名有哪些?
\begin{itemize}
    \item Linux:默认情况下,MySQL的数据文件存储在\verb|/var/lib/mysql/|目录下。每个数据库对应一个子目录,表文件则直接存储在这个子目录内。
    \item Windows:默认存储路径可能是\verb|C:\ProgramData\MySQL\MySQL Server X.X\Data\|,结构与Linux相似。
    \item MacOS:类似Linux,具体位置依赖于安装方式,默认可能在\verb|/usr/local/mysql/data/|。
\end{itemize}

关于文件扩展名:
\begin{itemize}
    \item .frm:表结构定义文件。
    \item .MYD:MyISAM存储引擎的数据文件。
    \item .MYI:MyISAM存储引擎的索引文件。
    \item .ibd:InnoDB存储引擎的表空间文件,每个表单独一个文件(当使用\verb|innodb_file_per_table|选项时)。
    \item .opt:数据库级别的选项文件。
\end{itemize}

\section{任务4:数据表的创建与修改管理实验}
\subsection{实验目的}
\begin{enumerate}
    \item 掌握表的基础知识。
    \item 掌握使用MySQL Workbench或其他第三方管理工具和SQL语句创建表的方法。
    \item 掌握表的修改、查看、删除等基本操作方法。
    \item 掌握表中完整性约束的定义。
    \item 掌握完整性约束的作用。
\end{enumerate}
\subsection{实验内容}
(一) 表定义与修改操作
在library数据库中创建一个 bookInfo 表,表结构如下:
\begin{table}[!htb]
    \centering
    \begin{tabular}{|c|c|c|c|c|c|c|c|}\hline
    字段名&字段描述&数据类型&主键&外键&非空&唯一&自增\\\hline
    id&编号&INT(4)&是&否&是&是&是\\\hline
    isbn&国际标准书号&VARCHAR(13)&否&否&是&是&否\\\hline
    title&书名&VARCHAR(100)&否&否&是&否&否\\\hline
    author&作者&VARCHAR(100)&否&否&是&否&否\\\hline
    \verb|publish_date|&出版日期&DATE&否&否&否&否&否\\\hline
    category&分类&VARCHAR(50)&否&否&是&否&否\\\hline    
    \end{tabular}
\end{table}
按照下列要求进行表定义操作:

(1)首先创建数据库library。(1分)\\
创建方法同任务三,在MySQL Workbench或其他第三方管理工具中,点击“创建数据库”按钮,输入名称“library”,点击确定。
 
(2)创建 bookInfo 表。(1分)\\
在右侧栏中点击刚刚创建完成的library数据库,右键下拉选项中的`Tables',选择`Create Table',
输入表名,按照要求键入每一列的属性以及约束条件,点击确定。
  
(3)将 bookInfo 表的title字段的数据类型改为 VARCHAR(150)。(1分)\\
在`tables'栏目下,右键刚刚建好的表`bookInfo',选择`Alter Table'。弹出的窗口界面与(2)相同,按照
修改title字段的`Datatype'为VARCHAR(150)。点击确定,可以看到MySQL自动帮我们生成了SQL语句。
 
(4)将\verb|publish_date|字段的位置改到author字段的前面。(1分)\\
同(3),在修改界面中右键该字段,选择`Move Up',将其移动到author字段的前面。点击确定,可以看到
MySQL自动帮我们生成了SQL语句。
 
(5)将isbn字段改名为\verb|book_isbn|。(1分)\\
同上,在修改界面双击isbn字段,修改名称为\verb|book_isbn|,点击确定,可以看到MySQL自动帮我们生成了SQL语句。 

(6)将 bookInfo 表的category字段删除。(1分)\\
同(3),在修改界面中右键`category',选择`Delete selected',点击确定,可以看到MySQL自动帮我们生成了SQL语句。
 
(7)在 bookInfo 表中增加名为price的字段,数据类型为DECIMAL(8,2)。(1分)\\
同上,在修改页面新增一列,输入列名为`price',数据类型为DECIMAL(8,2),点击确定,可以看到MySQL自动帮我们生成了SQL语句。

(8)将 bookInfo 表改名为 bookDetails。(1分)\\
同(3),在修改界面左上方输入新表名`bookDetails',可以看到MySQL自动帮我们生成了SQL语句。
 
(9)将 bookDetails 表的存储引擎更改为InnoDB类型。(1分)\\
在右上方的`Engine'下拉框中选择`InnoDB',点击确定。这里我默认的Engine为InnoDB。

上面操作流程对应的截屏\hyperref[fig:4]{如下:}
\begin{figure}[!htb]\label{fig:4}
    \centering
    \includegraphics[width=0.3\textwidth]{newpic/4-1.png}\hspace{0.5em}
    \includegraphics[width=0.3\textwidth]{newpic/4-2.png}\hspace{0.5em}
    \includegraphics[width=0.3\textwidth]{newpic/4-3.png}\vspace{1em}

    \includegraphics[width=0.3\textwidth]{newpic/4-4.png}\hspace{0.5em}
    \includegraphics[width=0.3\textwidth]{newpic/4-5.png}\hspace{0.5em}
    \includegraphics[width=0.3\textwidth]{newpic/4-6.png}\vspace{1em}

    \includegraphics[width=0.3\textwidth]{newpic/4-7.png}\hspace{0.5em}
    \includegraphics[width=0.3\textwidth]{newpic/4-8.png}\hspace{0.5em}
    \includegraphics[width=0.3\textwidth]{newpic/4-9.png}\vspace{1em}
\end{figure}

(二)创建图书馆员工管理数据库libraryStaff,并定义department表和staff表,完成两表之间的完整性约束。
按照下列要求进行表操作:(基本操作大同小异,这里我在每一个小问里面给出对应的SQL语句)
\begin{table}[!htb]
    \centering
    \begin{tabular}{|c|c|c|c|c|c|c|c|}
        \multicolumn{8}{c}{Department表的结构}
        \\\hline
        字段名&字段描述&数据类型&主键&外键&非空&唯一&自增\\\hline
        \verb|dept_id|&部门号&INT(4)&是&否&是&是&否\\\hline
        \verb|dept_name|&部门名&VARCHAR(50)&否&否&是&是&否\\\hline
        functions&部门职能&VARCHAR(100)&否&否&否&否&否\\\hline
        location&部门位置&VARCHAR(100)&否&否&否&否&否\\\hline
    \end{tabular}\vspace{1em}
    \begin{tabular}{|c|c|c|c|c|c|c|c|}        
        \multicolumn{8}{c}{Staff表的结构}\\\hline
        字段名&字段描述&数据类型&主键&外键&非空&唯一&自增\\\hline
        id&员工号&INT(4)&是&否&是&是&否\\\hline
        \verb|staff_num|&员工编号&VARCHAR(10)&否&否&是&是&否\\\hline
        \verb|dept_id|&部门号&INT(4)&否&是&否&否&否\\\hline
        name&姓名&VARCHAR(50)&否&否&是&否&否\\\hline
        sex&性别&VARCHAR(10)&否&否&否&否&否\\\hline
        \verb|birth_date|&出生日期&DATE&否&否&否&否&否\\\hline
        address&家庭住址&VARCHAR(100)&否&否&否&否&否\\\hline
    \end{tabular}
\end{table}

(1)创建libraryStaff数据库。(1分)
\begin{verbatim}
    CREATE DATABASE libraryStaff;
\end{verbatim}

(2)创建department表,并附上SQL语句。(1分)
\begin{verbatim}
    USE libraryStaff;
    CREATE TABLE department(
        dept_id INT(4) PRIMARY KEY,
        dept_name VARCHAR(50) NOT NULL,
        functions VARCHAR(100),
        loacation VARCHAR(100),
        CONSTRAINT unique_dept UNIQUE(dept_name)
    );
\end{verbatim}

(3)创建staff表(注意外键),并附上SQL语句。(1分)
\begin{verbatim}
    USE libraryStaff;
    CREATE TABLE staff(
        id INT(4) PRIMARY KEY AUTO_INCREMENT UNIQUE KEY,
        staff_num VARCHAR(10) NOT NULL,
        dept_id INT(4), 
        name VARCHAR(50) NOT NULL,
        sex VARCHAR(10) NOT NULL,
        birth_date DATE,
        address VARCHAR(100),
        CONSTRAINT unique_staff UNIQUE(staff_num),
        CONSTRAINT foreign_key_staff
        FOREIGN KEY(dept_id) REFERENCES department(dept_id)
    );
\end{verbatim}
 
(4)删除department表,并附上SQL语句。(1分)
\begin{verbatim}
    USE libraryStaff;
    DROP TABLE department;
\end{verbatim}
欸,但是此时我们发现右下方的输出栏提示我们``Cannot drop table `department'
referenced by a foreign key constraint `\verb|foreign_key_staff|' on the table `staff'"
也就是由于外键的存在,删除失败了!说明在删除department表之前,我们需要先删除staff
表的外键约束,否则将导致剩下的`staff'表出问题!
\begin{figure}[!htb]
    \centering
    \includegraphics[width=0.8\textwidth]{newpic/4-2-1.png}
\end{figure}

(5)删除staff表的外键约束,代码如下:(1分)
\begin{verbatim}
    USE libraryStaff;
    ALTER TABLE staff DROP FOREIGN KEY foreign_key_staff;
\end{verbatim}
 
(6)重新删除department表,并附上SQL语句。(1分)
\begin{verbatim}
    USE libraryStaff;
    DROP TABLE department;
\end{verbatim}
这时我们发现,右下方输出信息提示成功删除!
\begin{figure}[!htb]
    \centering
    \includegraphics[width=0.8\textwidth]{newpic/4-2-2.png}
\end{figure}

\subsection{观察与思考}
1、关于NOT NULL 

(1) 在定义基本表语句时,NOT NULL参数的作用是什么?(1分)\\
NOT NULL参数的作用是约束该字段不能为NULL,即该字段的值必须非空。
在定义某字段为NOT NULL时,如果更新或者插入的目标会使某一行的该字段为空
数据库会报错并拒绝执行该操作。这时常见的数据完整性约束之一。

(2)主码列修改成允许NULL能否操作?为什么?(2分)\\
不允许。主码(Primary Key)是用来唯一标识表中每一行记录的字段或字段组合。
主码的两大特性是唯一性和非空性就保证了主码一定不能为NULL,否则会导致数据库出现错误。

2、关于外码

(1) 根据下面设计的表结构,Staff表的外键能否设置成功?思考外码设置需要注意哪些问题?(3分)\\
\begin{table}[!htb]
    \centering
    \begin{tabular}{|c|c|c|c|c|}
    \multicolumn{5}{c}{Department表的结构} \\\hline
    字段名 & 字段描述 & 数据类型 & 主键 & 外键 \\\hline
    \verb|dept_no| & 部门号 & INT(4) & & \\\hline
    \verb|dept_name|&部门名&VARCHAR(50)&&\\\hline
    \end{tabular}

    \vspace{2em}
    \begin{tabular}{|c|c|c|c|c|}
    \multicolumn{5}{c}{Staff表的结构} \\\hline
    字段名 & 字段描述 & 数据类型 & 主键 & 外键 \\\hline
    \verb|staff_no| & 员工编号 & VARCHAR(10) & & \\\hline
    \verb|dept_no|&所在部门号&INT(4)&&√\\\hline
    name&姓名&VARCHAR(50)&&\\\hline
    \end{tabular}
\end{table}
不能成功,因为\verb|dept_no|字段在Staff表中定义时并未被设置成主键,而外键的定义依赖于主键。

外码的设置需要注意以下几点:
\begin{enumerate}
    \item 是依赖于数据库中已存在的表的主键。
    \item 数据类型和长度必须匹配: 外键列与目标列的数据类型、长度、字符集等必须完全一致。
    \item 外键列不能有主表的目标列中没有的值。
    \item 注意设置级联操作以保证数据一致性
\end{enumerate}

(2) 如果主表无数据,从表的数据能输入吗?(1分)\\
不行。外键约束要求从表的外键列中的值必须存在于主表的目标列中。
如果主表中没有数据,从表的外键列就无法引用有效的值,因此插入操作会失败。
但是在从表该列未被设置为NOT NULL时,可以输入NULL值。

(3) 先创建从表,再创建主表是否可以?(1分)\\
不可以。在创建从表时,如果设置了外键约束,数据库会检查主表是否存在以及主表的目标列是否有效。
如果主表尚未创建,从表的外键约束将无法验证,导致创建失败。

3、关于主码和唯一约束

(1) 唯一约束列是否允许NULL值?(1分)\\
唯一约束列允许NULL值,并且在同一个表格中可以有多个行为NULL值。但该字段不为NULL的行之间的值一定不相同。

(2) 一张表可以设置几个主码,可以设置几个唯一约束?(1分)\\
一张表只能设置一个主码,但可以设置多个唯一约束。

\end{document}